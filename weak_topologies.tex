\chapter{Topologie deboli}
Ricordiamo che in spazi di Banach a dimensione infinita, non vale la caratterizzazione dei compatti come insiemi chiusi e limitati.

\begin{theorem}
\label{th:unit_ball_not_compact}
	In uno spazio normato di dimensione infinita la palla unitaria chiusa non è compatta.
\end{theorem}

Il teorema fa leva su un noto lemma di Riesz:

\begin{lemma}[Riesz]
\label{lemma:riesz}
	Sia $E$ normato, $V \lneq E$ sottospazio finito-dimensionale.
	Allora esiste $z \in E$ di norma unitaria e tale che $d(z, V) \geq 1$.
\end{lemma}
\begin{proof}
	Sia $x \in E \setminus V$. Il singoletto $\{x\}$ è compatto, mentre $V$ è chiuso, dunque, dal momento che $E$ è di Hausdorff, $d(x,V) = r > 0$.
	Sia $C = V \cap \closure B(x, 2r)$. Per compattezza di $C$ in $V$ ($C$ è infatti chiuso e limitato nello spazio finito-dimensionale $V$), la distanza $d(x,V)$ è realizzata da un punto $y^* \in C \subseteq V$, i.e.\ $r = \|x-y^*\|$. Poniamo $z = (x-y^*)/r$ e verifichiamo che soddisfa $d(z, V) \geq 1$: infatti, per ogni $v \in V$
	\begin{equation*}
		\|z-v\| = \left\| \frac{x-y^*}{\|x-y^*\|} - v \right\| = \frac{\|x-\overbrace{y^* - v\|x-y^*\|}^{\in V}\|}{\|x-y^*\|} \geq \frac{r}r =1.
	\end{equation*}
\end{proof}

\begin{proof}[Dimostrazione del Teorema~\ref{th:unit_ball_not_compact}]
	Definiamo una successione per ricorsione, partendo da un qualsiasi $z_1$ di norma unitaria. Per il lemma di Riesz, esiste $z_2 \in \partial B(0,1)$ tale che $\|z_2 - z_1\| \geq 1$ e $z_2 \notin \langle z_1 \rangle$.
	Iterando questo procedimento, si ottiene una successione $\{z_n\}_{n \in \N}$ tale che
	\begin{equation*}
		\|z_n\| = 1, \qquad z_n \notin \langle z_1, \ldots, z_{n-1}\rangle, \qquad \|z_n - z_m\| \geq 1,\ \text{per ogni $n, m \in \N$}.
	\end{equation*}
	La quale testimonia la non compattezza della palla unitaria.
\end{proof}

Avere pochi compatti è una grande ostruzione! Nella pratica matematica infatti, è comodo poter invocare teoremi come quello di Weierstrass:

\begin{theorem}[Weierstrass]
	Sia $f:X \to \R$ una funzione continua sullo spazio topologico $X$.
	Se $K \subseteq X$ è compatto, $f$ ammette massimo e minimo su $K$.
\end{theorem}

Ciò si usa, ad esempio, in calcolo delle variazioni per ottenere l'esistenza di minimi/massimi di funzionali.

\section{Topologie iniziali}
Si può artificialmente aumentare il numero di compatti togliendo aperti dalla topologia. Infatti questo diminuisce i possibili ricoprimenti di un chiuso, così che `sia più facile' per un insieme soddisfare la proprietà di compattezza.

Chiaramente questo va equilibrato con l'esigenza di mantenere una topologia rilevante allo studio dell'analisi funzionale. Il `tipping point' è dato da $E'$: togliendo aperti infatti diminuiscono anche le possibili funzioni continue su $E$. Per cui prendiamo la topologia meno fine che conserva la continuità di tutti i funzionali su $E$ (definiti rispetto alla topologia naturale), ossia la topologia iniziale della famiglia $E'$.

\begin{definition}
	Data una famiglia di funzioni $\{\varphi_i : X \to Y_i\}_{i \in I}$, dove ciascun $Y_i$ è equipaggiato con una topologia, si chiama \defining{topologia iniziale} la topologia meno fine su $X$ che rende continue tutte le $\varphi_i$
\end{definition}

Se $\varphi : X \to Y$ è una funzione il cui codominio è dotato di una topologia, denotiamo con $\langle \varphi \rangle$ la topologia generata dalle controimmagini degli aperti di $Y$ secondo $\varphi$. Allora la topologia iniziale associata alla famiglia $\{\varphi_i : X \to Y_i\}_{i \in I}$ è data da
\begin{equation*}
	\tau := \bigcap_{i \in I} \langle \varphi_i \rangle.
\end{equation*}
Più esplicitamente, $\tau$ contiene tutte le unioni di controimmagini di insiemi aperti rispetto alle $\varphi_i$.

Ricordiamo che una base di intorni per $x_0 \in X$ topologico è una famiglia $\{A_j\}_{j \in J}$ di intorni aperti di $x_0$ tali che ogni intorno di $x_0$ contiene almeno un $A_j$.

Nel caso della topologia iniziale indotta da una famiglia $\{\varphi_i : X \to Y_i\}_{i \in I}$, una base di intorni è data da
\begin{equation}
\label{eq:neigh_basis}
	\left\{ {\textstyle \bigcap_{i \in H} \varphi_i^{-1}(A_i)} \suchthat \text{$H \subseteq I$ finito e per ogni $i \in H$, $A_i$ aperto e $\varphi_i(x_0) \in A_i$} \right\}
\end{equation}

\begin{theorem}
\label{th:weak_conv_and_cont}
	Siano $X$, $Y_i$ e $\{\varphi_i : X \to Y_i\}_{i \in I}$ una situazione come sopra. Si doti $X$ della topologia iniziale indotta dalla famiglia $\{\varphi_i\}_{i \in I}$.
	Allora
	\begin{enumerate}
		\item Una successione $\{x_n\}_{n \in \N}$ in $X$ converge a $x \in X$ se e solo se per ogni $i \in I$, $\varphi_i(x_n) \conv[n] \varphi_i(x)$ in $Y_i$.
		\item Sia $Z$ uno spazio topologico e $\psi : Z \to X$ una funzione.
		Allora $\psi$ è continua se e solo se, per ogni $i \in I$, $\varphi_i\psi$ è continua.
	\end{enumerate}
\end{theorem}
\begin{proof}
	\leavevmode
	\begin{enumerate}
		\item Supponiamo che $x_n \conv[n] x$ in $X$. Per continuità, le $\varphi_i$ conservano la convergenza. D'altra parte, supponiamo che ogni $\{\varphi_i(x_n)\}_{n \in \N}$ converga a $\varphi_i(x)$. Fissiamo un intorno di base di $x$, cioè un intorno aperto dato dall'intersezione di un numero finito di controimmagini della $\varphi_i$:
		\begin{equation*}
			x \in U=\bigcap_{i \in H} \varphi_i^{-1}(A_i), \ \text{$H \subseteq I$ finito. p.o. $i \in H$, $A_i$ aperto, $\varphi_i(x_0) \in A_i$}.
		\end{equation*}
		Fissato $i \in H$, per ipotesi di convergenza, tutti i punti $\varphi_i(x_n)$ da un certo $n_i \in \N$ in poi cadranno in $A_i$, perciò $x_n \in \varphi_i^{-1}(A_i)$. Preso $\bar n = \max \{n_i \suchthat i \in H\}$, deduciamo che da $\bar n$ n avanti tutti i punti $x_n$ giacciono in $U$, ossia $x_n \conv[n] x$.
		\item Chiaramente se $\psi$ è continua si ha che $\varphi_i \psi$ è continua per ogni $i \in I$. D'altra parte, supponiamo che $\varphi_i \psi$ sia continua per ogni $i \in I$. Fissiamo $A \in X$ di base, cioè
		\begin{equation*}
			A = \bigcap_{i \in H} \varphi_i^{-1}(A_i), \qquad \text{$H \subseteq I$ finito e per ogni $i \in H$, $A_i$ aperto}
		\end{equation*}
		Per ipotesi, $(\varphi_i \psi)^{-1}(A) = \psi^{-1}(\varphi_i^{-1}A)$ è aperto, ma questo significa esattamente che $\psi$ è continua.
	\end{enumerate}
\end{proof}

\begin{example}
	La topologia prodotto è iniziale rispetto alla famiglia delle proiezioni.
\end{example}

\section{Topologia debole}
\begin{definition}
	Sia $E$ uno spazio normato.
	La \defining{topologia debole} di $E$ è la topologia iniziale su $E$ rispetto alla famiglia di mappe $E'$.
\end{definition}

\begin{remark}
	La topologia debole su $E$ è indicata con $\sigma(E, E')$.
\end{remark}

Un sottoinsieme di $E$ si dice \defining{debolmente chiuso} (risp. aperto) se lo è rispetto alla topologia debole su $E$. Analogamente, parliamo di successioni \defining{debolmente convergenti}, e scriviamo $x_n \weakconv x$ per indicarlo.

Sia $x_0 \in E$. Una base di intorni per $x_0$ rispetto alla topologia debole è data come nel caso generale dalle intersezione finite di controimmagini di aperti contenenti l'immagine di $x_0$, rispetto a funzionali lineari.
Nel nostro caso specifico però, possiamo prendere gli aperti su $\R$ come intervalli aperti, da cui l'espressione generica per un intorno fondamentale:
\begin{equation*}
	\{x \in E \suchthat |f(x - x_0)| < \varepsilon, \text{per ogni $f \in H$} \}, \quad \text{per $H \subset E'$ finito}.
\end{equation*}

\begin{theorem}
	Nel caso $E$ sia uno spazio normato di dimensione finita, la topologia debole coincide con la topologia indotta dalla norma di $E$.
\end{theorem}
\begin{proof}
	È sufficiente mostrare che un aperto rispetto alla norma è anche debolmente aperto, o ancora più semplicemente, dati $x_0 \in E$ e $r > 0$, dobbiamo mostrare che esiste un intorno debole $U$ di $x_0$ contenente $B_E(x_0, r)$.

	Sia $e_1, \ldots, e_n$ una base di $E$. Consideriamo le proiezioni canoniche $\pi_i : E \to \R$, che mandano un vettore $x \in E$ nella sua $i$-esima componente rispetto alla base scelta. Queste sono lineari e continue, dunque $\pi_i \in E'$ per ogni $1 \leq i \leq n$.
	Sia $U$ l'intorno debole del punto $x_0$ così definito:
	\begin{eqalign*}
		U &= \{ x \in E \suchthat |\pi_i(x-x_0)| < \varepsilon, \text{per ogni $1 \leq i \leq n$} \}, \qquad \varepsilon > 0\\
		&= \{ x \in E \suchthat |x_i-{x_0}_i| < \varepsilon, \text{per ogni $1 \leq i \leq n$} \}\\
		&= \{ x \in E \suchthat \max_{1 \leq i \leq n} |x_i-{x_0}_i| < \varepsilon \}\\
		&= B_{\|\cdot\|_\infty}(x_0, \varepsilon).
	\end{eqalign*}
	Siccome in dimensione finita tutta le norme sono equivalenti, esiste un $\varepsilon > 0$ tale che $B_{\|\cdot\|_\infty}(x_0, \varepsilon) \subseteq B(x_0, r)$, come volevasi dimostrare.
\end{proof}
\begin{theorem}
	Nel caso $E$ sia uno spazio normato di dimensione infinita, la topologia debole è strettamente meno fine della topologia indotta dalla norma di $E$, in particolare ogni intorno debole $U$ di ogni punto $x_0 \in E$, esiste sempre una retta (sottospazio affine di $E$ di dimensione $1$) contenuta in $U$.
\end{theorem}
\begin{proof}
	Prendiamo un qualsiasi intorno di base di un punto $x_0$:
	\begin{equation*}
		U = \{ x \in E \suchthat |f_i(x-x_0)| < \varepsilon, \text{per ogni $1 \leq i \leq n$} \}, \ n \in \N,\ \varepsilon > 0,\ f_i, \ldots, f_n \in E'.
	\end{equation*}
	Sia $F: E \to \R^n$ la mappa $x \mapsto (f_1(x), \ldots, f_n(x))$. Siccome $\dim E = \infty$, $F$ non è iniettiva (altrimenti $\dim E \leq n$). Dunque $\ker F \neq 0$, ossia esiste un vettore non nullo $y_0 \in E$ che annulla tutte le $f_i$ simultaneamente. Allora consideriamo la retta $x_0 + t y_0$, parametrizzata da $t \in \R$. Si verifica che per ogni $t \in \R$:
	\begin{equation*}
		f_i(x_0+ty_0 - x_0) = t \cancelto{0}{f_i(y_0)}=0 < \varepsilon,
	\end{equation*}
	cioè la retta è contenuta in $U$, come volevasi dimostrare.
\end{proof}

\begin{remark}
	Dalla dimostrazione di questo teorema vediamo che ci sono in realtà parecchie rette contenute in un qualsiasi intorno, poichè $\ker F$ ha codimensione finita in uno spazio a dimensione infinita.
\end{remark}

Nonostante in dimensione finita la topologia debole sia uguale a quella usuale, è interessante andare a vedere come sono gli intorni fondamentali. Si vede che essi sono costruiti per `taglio' dello spazio con iperpiani, e quindi in dimensione finita possono delimitare regioni limitate di spazio di forma poliedrale.
In dimensione infinita ciò non è possibile, perchè intuitivamente non ho mai abbastanza iperpiani per chiudere il poliedro (ne servirebbero infiniti ma ne posso utilizzare solo in numero finito).

\begin{corollary}
	Se la dimensione di $E$, spazio normato, è infinita, la \emph{sfera} unitaria $S$ non è debolmente chiusa.
\end{corollary}
\begin{proof}
	Sia $x_0 \in E$ di norma inferiore a $1$. Sia $U$ un intorno debole di $x_0$. Sia $R$ una retta contenuta in $U$ e diretta da $y_0$. Si ponga $g(t) = \|x_0 + ty_0\|$, e si osservi che $g$ è continua su $R$. D'altra parte $g(0) = \|x_0\| < 1$ mentre $g(+\infty) = + \infty$, per cui deve esistere almeno un punto di $R$ che giace al di fuori della palla, e quindi $R$ interseca $S$ in almeno un punto.

	Questo dimostra che $S$ non è chiuso, siccome $E \setminus S$ dovrebbe essere aperto e contenere un $U$, ma in $U$ giacciono punti di $S$, assurdo.
\end{proof}

\begin{exercise}
	Qual è la chiusura debole di $S$?
\end{exercise}

Per svolgere l'esercizio, è utile il seguente:

\begin{theorem}
	Sia $E$ normato, sia $C$ sottoinsieme di $E$ chiuso e convesso.
	Allora $C$ è debolmente chiuso.
\end{theorem}
\begin{proof}[Idea delle dimostrazione]
	Si usa il seguente teorema di separazione:
	\begin{theorem}[Hahn--Banach geometrico]
	\label{th:geom_hahn_banach}
		Sia $E$ normato, siano $C$ chiuso e $K$ compatto in $E$, disgiunti e convessi.
		Allora sono separati da un iperpiano, cioè esiste $f \in E$' ed esistono $\alpha, \beta \in \R$ tali che
		\begin{equation*}
			f(x) \leq \alpha < \beta \leq f(y), \qquad \text{per ogni $x \in C$, $y \in K$.}
		\end{equation*}
	\end{theorem}
	Sapendo ciò, ci poniamo nel caso non banale $C \neq E$. Prendiamo un punto $x_0 \in E \setminus C$. Il singoletto $\{x_0\}$ è compatto e convesso, dunque per il teorema di Hahn--Banach geometrico possiamo separarli con un iperpiano, cioè esistono $f \in E$' ed $\alpha, \beta \in \R$ tali che
	\begin{equation*}
		f(x) \leq \alpha < \beta \leq f(x_0), \qquad \text{per ogni $x \in C$}.
	\end{equation*}
	Perciò $x_0 \in f^{-1}(\alpha, \infty) = U \in \sigma(E, E')$, e $U \subseteq E \setminus C$. Ma allora $E \setminus C$ è un aperto debole, in quanto contiene un intorno debole di $x_0$.
\end{proof}

\begin{lemma}
	La topologia debole è di Hausdorff, cioè se $x_1, x_2 \in E$ sono punti distinti allora esistono due aperti deboli disgiunti $U_1$ e $U_2$ tali che $x_1 \in U_1$ e $x_2 \in U_2$.
\end{lemma}
\begin{proof}
	Per il Corollario~\ref{cor:two_chap_1} al teorema di Hahn--Banach, essendo $x_1 - x_2 \neq 0$ per ipotesi, esiste un operatore lineare continuo $f \in E'$ tale che $f(x_1 - x_2) = \|x_1 - x_2\| \neq 0$. Allora poniamo $\varepsilon = |f(x_1-x_2)|/4$, e si prendano gli intorni deboli di $x_1$ e $x_2$ dati rispettivamente da:
	\begin{equation*}
		U_1 = \{x \in E \suchthat |f(x-x_1)| < \varepsilon \}, \qquad U_2 = \{x \in E \suchthat |f(x-x_2)| < \varepsilon \}.
	\end{equation*}
	Dichiariamo che $U_1 \cap U_2 = \varnothing$. Infatti, per assurdo, sia $x \in U_1 \cap U_2 \neq \varnothing$. In tal caso si avrebbe
	\begin{eqalign*}
		|f(x_1- x_2)| &\leq |f(x_1 - x + x - x_2)|\\
		&\leq |f(x_1-x)| + |f(x-x_2)|\\
		&< 2\varepsilon = \frac{|f(x_1-x_2)|}2,
	\end{eqalign*}
	impossibile.
\end{proof}

\begin{remark}
	Il lemma ha la conseguenza estremamente importante di garantire l'unicità del limite debole.
\end{remark}

\begin{lemma}
\label{lemma:weaktop_five}
	Sia $E$ uno spazio normato, $\{x_n\}_{n \in \N}$ una successione di punti di $E$. Allora
	\begin{enumerate}
		\item $x_n \weakconv x$ se e soltanto se $f(x_n) \conv f(x)$ per ogni $f \in E'$,
		\item $x_n \conv x$ comporta $x_n \weakconv x$,
		\item $x_n \weakconv x$ comporta che $\{x_n\}_{n \in \N}$ è limitata, ed in particolare
		\begin{equation*}
			\|x\|_E \leq \liminf_n \|x_n\|_E
		\end{equation*}
		\item Se $x_n \weakconv x$ e $f_n \conv f$ in $E'$, allora $f_n(x_n) \conv f(x)$.
	\end{enumerate}
\end{lemma}
\begin{proof}
	\leavevmode
	\begin{enumerate}
		\item Si veda il Teorema~\ref{th:weak_conv_and_cont}.
		\item Banalmente, siccome la topologia debole è meno fine.
		\item Sia $J_y : E' \to \R$ la valutazione su $y \in E$. Vale $\|J_y\|_{E''} = \|y\|_E$ (Corollario~\ref{cor:eval_norm}), quindi si ha che $x_n \weakconv x$ implica $J_{x_n} \conv J_x$ puntualmente. Si conclude applicando il Corollario~\ref{cor:banach_steinhaus_liminf} del teorema di Banach--Steinhaus.
		\item Si ha
		\begin{eqalign*}
			|f_n(x_n) - f(x)| &\leq |f_n(x_n) - f(x_n)| + |f(x_n) - f(x)|\\
			&\leq \|f_n-f\|_{E'}\,\|x_n\|_E + |f(x_n) - f(x)|\\
			&\leq \underbrace{\|f_n-f\|_{E'}}_{\conv 0} (\liminf_n \|x_n\|_E) + \underbrace{|f(x_n) - f(x)|}_{\conv 0}\\[-1.5ex]
			&\conv 0.\\[-1ex]
		\end{eqalign*}
	\end{enumerate}
\end{proof}

\begin{remark}
	Sottolineiamo il seguente fatto che abbiamo impiegato nella dimostrazione precedente:
	\begin{equation*}
		x_n \weakconv x \sse J_{x_n} \conv J_x\ \text{puntualmente,}
	\end{equation*}
	In questo senso, la convergenza debole in $E$ è la convergenza puntuale in $E''$.
	Questo legame tra $E$ ed $E''$ verrà approfondito più avanti in questo capitolo.
\end{remark}

\begin{theorem}
	Siano $E$ ed $F$ spazi di Banach, $T : E \to F$ lineare. Allora $T$ è continua rispetto alle topologie delle norme se e solo se è continua rispetto alle topologie deboli.
\end{theorem}
\begin{proof}
	Supponiamo $T$ sia continua rispetto alle norme. Dal Teorema~\ref{th:weak_conv_and_cont}, per provare che $T$ è continua rispetto alle topologie deboli basta provare che $\varphi T : (E, \sigma(E,E')) \to \R$ è continua, per ogni $\varphi \in F'$. Ma questo è vero dal momento che $\varphi T \in E'$.

	Viceversa, supponiamo che $T$ sia continua rispetto alle topologie deboli. Abbiamo dimostrato che $(F, \sigma(F,F'))$ è Hausdorff, il che ci permette di dire che $G(T)$ è chiuso (per proprietà generali degli spazi di Hausdorff).
	Consideriamo allora la mappa
	\begin{eqalign*}
		I : (E, \|\cdot\|_E) \times (F,\|\cdot\|_F) &\longto (E, \sigma(E,E')) \times (F, \sigma(F,F'))\\
			(x,y) &\longmapsto (x,y)
	\end{eqalign*}
	Si osserva che $I$ è continua, sicchè la topologia al codominio è meno fine della topologia al dominio.
	Pertanto $I^{-1}(G(T)) = G(T)$ è chiuso, da cui segue (teorema del grafico chiuso) che $T$ è continua rispetto alla topologia delle norme.
\end{proof}

\begin{remark}
	Il teorema ci dice che passare alla topologia debole non ingrandisce lo spazio duale. Si osservi, d'altra parte, che l'ipotesi di linearità è cruciale. Ad esempio la norma di $E$ è continua rispetto alla topologia che induce, ma non debolmente continua: abbiamo mostrato che $\|\cdot\|^{-1}\{1\} = S$ non è debolmente chiuso.
\end{remark}
\begin{remark}
	Senza l'ipotesi di completezza abbiamo comunque che la forte continuità di un operatore $T:E \to F$ ne implica la debole continuità, tuttavia il contrario non è più garantito.
\end{remark}

\begin{exercise}
	Sia $E=\Czero[0,1]$, dotato della norma $\|\cdot\|_\infty$, con l'operatore $Tx = \int_0^1 x^2(t)\,\dt$. Provare che
	\begin{enumerate}
		\item $T$ è fortemente continua,
		\item $T$ non è debolmente continua,
		\item Se $x_n \weakconv x$ in $E$, allora $Tx_n \conv Tx$, cioè che $T$ è sequenzialmente debolmente continua.
		\item Dedurre che $E$, dotato della topologia debole, non è metrizzabile.
	\end{enumerate}

	\textbf{Svolgimento}.
	\begin{enumerate}
		\item L'integrale rispetta la convergenza uniforme, quindi se $x_n \conv x$ in $E$ allora anche $x^2_n \conv x^2$ e dunque $Tx_n \conv Tx$.
		\item Se $T$ fosse debolmente continua, allora $T^{-1}(-1,1)$ sarebbe debolmente aperto.
		Si osservi che la funzione nulla è contenuta in $T^{-1}(1,1)$, poichè il suo integrale è nullo.
		Ora, se $T^{-1}(-1,1)$ fosse aperto, conterrebbe un'intera retta passante per $0$, diretta diciamo da un vettore $v \in E$ non nullo.
		D'altra parte questo vorrebbe dire che $T(0+sv) = s^2 \int_0^1 v(t)\,\dt \in (-1,1)$ per ogni $s \in \R$, assurdo.
		\item Siccome le valutazioni puntuali $v_s : E \to \R$ ($v_s(x) = x(s)$) sono nel duale $E'$, la convergenza debole delle $x_n$ ci permette di concludere la convergenza $v_s(x_n) \conv v_s(x)$, cioè che $x_n \conv x$ puntualmente.
		Inoltre, il fatto che $x_n \weakconv x$ implica che $\|x_n\|_\infty \leq M$ uniformemente in $n \in \N$ (Lemma~\ref{lemma:weaktop_five}).
		Allora $Tx_n \conv Tx$ per convergenza dominata.
		\item Se fosse metrizzabile, la continuità per successioni sarebbe equivalente alla continuità per aperti, in conflitto con il punto precedente.
	\end{enumerate}
\end{exercise}

Si osservi che la continuità di $T$ è fallita perchè siamo stati in grado di scegliere un aperto limitato in $\R$. Se equipaggiassimo $\R \cup \{+\infty\}$ con la topologia inferiore (la topologia generata dalle semirette destre aperte), ciò non sarebbe possibile.

Ricordiamo che in questa topologia, $x_n \conv x$ se e solo se $x \leq \liminf_n x_n$. In particolare, tale topologia non è di Hausdorff!
Una mappa a valori in $\R$ che sia continua rispetto alla topologia inferiore è detta \defining{inferiormente semicontinua}.

\begin{theorem}
\label{th:weaktop_seven}
	Sia $E$ normato, $F : {E \to (-\infty, +\infty]}$ convessa.
	Se ${F : (E, \|\cdot\|_E) \to \R}$ è inferiormente semicontinua, allora $F : (E, \sigma(E,E')) \to \R$ è inferiormente semicontinua.
	In particolare, se $x_n \weakconv x$ allora $F(x) \leq \liminf_n F(x_n)$.
\end{theorem}
\begin{proof}
	Sia $C \subseteq (-\infty, + \infty]$ chiuso, cioè $C=(-\infty, M]$. Per convessità di $F$, $F^{-1}(C)$ è convesso\footnote{Si usa il fatto che presi due punti qualunque $x,y \in E$, $F(x+t(y-x)) \leq F(x) + t(F(y) - F(x))$, per ogni $t \in [0,1]$. Si noti che $C$ deve avere la forma di una semiretta sinistra affinchè questa disuguaglianza sia utile.}.
	Dacchè $F$ è topologica, si ha che $F^{-1}(C)$ è fortemente chiuso. Ma fortemente chiuso e convesso implica debolmente chiuso, quindi il teorema è dimostrato.
\end{proof}

\begin{remark}
	Dal teorema segue che la funzione $T: \Czero[0,1] \to (-\infty,+\infty]$ data dall'integrale del quadrato è debolmente continua.
\end{remark}

\begin{exercise}
	Sia $p \in (1, \infty)$, sia $\{x^{(n)}\}_{n \in \N}$ una successione di $\ell^p$.
	Provare che
	\begin{equation*}
		\text{$x^{(n)} \weakconv x$ se e solo se $\sup_n \|x^{(n)}\|_{\ell^p} < \infty$ e $x^{(n)}_h \conv x_h$ per ogni $h \in \N$}.
	\end{equation*}

	\textbf{Svolgimento}.
	\begin{description}
		\item[$(\Longrightarrow)$] Dal Lemma~\ref{lemma:weaktop_five}, la debole convergenza dà uniforme limitatezza. Inoltre
		\begin{equation*}
			{x^{(n)}_h = \pi_h(x^{(n)})},
		\end{equation*}
		ed essendo le $\pi_h$ lineari continue preservano la convergenza.

		\item[$(\implied)$] Dobbiamo provare che per ogni $\varphi \in (\ell^p)'$ si ha $\varphi(x^{(n)}) \conv \varphi(x)$.
		D'altra parte $(\ell^p)' \iso \ell^q$ per $q$ H\"older coniugato di $p$ (Esercizio~\ref{ex:ellp_iso_ellq}).
		In particolare, esiste $z \in \ell^q$ tale che $\varphi(y) = \sum_{h=1}^\infty x_h z_h$. Dobbiamo dunque verificare che
		\begin{equation*}
			\sum_{h=1}^\infty x^{(n)}_h z_h \conv \sum_{h=1}^\infty x_h z_h.
		\end{equation*}
		Posto $M = \sup_n \|x^{(n)}\|$, abbiamo
		\begin{eqalign*}
			&\left| \sum_{h=1}^\infty x^{(n)}_h z_h - \sum_{h=1}^\infty x_h z_h \right|\\
			&\leq \sum_{h=1}^\infty |x^{(n)}_h-x_h||z_h|\\
			&\leq \sum_{h=1}^N |x^{(n)}-x_h||z_h| + \sum_{h=N+1}^\infty |x^{(n)}-x_h||z_h|\\
			&\underset{\text{H\"o}}\leq \left( \sum_{h=1}^N |x^{(n)}_h-x_h|^p \right)^{1/p}\! \left( \sum_{h=1}^N |z_h|^q \right)^{1/q}\! +\\
			&\qquad + \left( \sum_{h=N+1}^\infty |x^{(n)}_h-x_h|^p \right)^{1/p}\! \left( \sum_{h=N+1}^\infty |z_h|^q \right)^{1/q}\\
			&\leq \left( \sum_{h=1}^N |x^{(n)}_h-x_h|^p \right)^{1/p}\!\! \|z\|_{\ell^q} + \left( \|x^{(n)}\|_{\ell^p} + \|x\|_{\ell^p} \right) \left( \sum_{h=N+1}^\infty |z_h|^q \right)^{1/q}\\
			&\leq (2M + \|z\|_{\ell^q}) \Bigg(\underbrace{\left( \sum_{h=1}^N |x^{(n)}_h-x_h|^p \right)^{1/p}}_{(1)} + \underbrace{\left( \sum_{h=N+1}^\infty |z_h|^q \right)^{1/q}}_{(2)} \Bigg)
		\end{eqalign*}
		Ora si fissi $\varepsilon > 0$, e sia $N$ preso in maniera che $(2) < \varepsilon$. Allora per puntuale convergenza delle $x^{(n)}$ a $x$, preso $n$ sufficientemente grande anche $(1) < \varepsilon$.
		Segue la convergenza cercata.
	\end{description}
\end{exercise}

\begin{remark}
	Nel precedente, $(\Longrightarrow)$ vale anche nel caso $p=1,\infty$, ma non $(\implied)$. Infatti nel caso $p=1,\infty$ il coniugato $q$ è rispettivamente $\infty$, $0$, il che non ci permette di portare avanti le stime utilizzate.
\end{remark}

\begin{theorem}
	Una successione $\{x^{(n)}\}_{n \in \N}$ di $\ell^1$ è fortemente convergente se e solo se è debolmente convergente.
\end{theorem}
\begin{proof}
	Omissis, vedere \cite{brezis2010functional}.
\end{proof}

\begin{remark}
	Quest'ultimo risultato mostra che la topologia debole su $\ell^1$ non è metrizzabile.
\end{remark}

\begin{remark}
	Nel caso $p=\infty$ si può rimpiazzare $\ell^\infty$ con $c_0$ come segue:
\end{remark}

\begin{exercise}
	Provare che $x^{(n)} \weakconv x$ in $c_0$ se e soltanto se
	\begin{equation*}
		\sup_n \|x^{(n)}\|_{\infty} < \infty \ \text{e}\ x^{(n)}_h \conv x_h \quad \text{per ogni $h \in \N$}.
	\end{equation*}
\end{exercise}

Riassumendo:

\begin{center}
	\begin{tabular}{cl}
		\textbf{Spazio} & \textbf{Convergenza debole se e solo se}\\
		$(\ell^1, \|\cdot\|_{\ell^1})$ & $x^{(n)} \conv x$\\
		$(\ell^p, \|\cdot\|_{\ell^p})$ & $\sup_n \|x^{(n)}\|_{\ell^p} < \infty \ \text{e}\ x^{(n)}_h \conv x_h \ \forall h \in \N.$\\
		$(c_0, \|\cdot\|_\infty)$ & $\sup_n \|x^{(n)}\|_{\infty} < \infty \ \text{e}\ x^{(n)}_h \conv x_h \ \forall h \in \N.$\\
		$(c, \|\cdot\|_\infty)$ & $\sup_n \|x^{(n)}\|_{\infty} < \infty \ \text{e}\ x^{(n)}_h \conv x_h \ \forall h \in \N \ \text{e}\ \lim_n\lim_h x^{(n)}_h = \lim_h \lim_n x_h^{(n)}$
	\end{tabular}
\end{center}
\vspace{1ex}

\begin{exercise}
	Sia $x \in \ell^p$ non nullo, $1 \leq p < \infty$. Consideriamo la successione:
	\begin{equation*}
		x^{(n)}_h = \begin{cases}
			0 & h <n\\
			x_{h-n+1} & h \geq n.
		\end{cases}
	\end{equation*}
	Discuterne la convergenza debole e forte.

	\textbf{Svolgimento}. Si osservi che $\|x^{(n)}\|_{\ell^p} = \|x\|_{\ell^p} \neq 0$, per ogni $n \in \N$. Consideriamo prima il caso $p \neq 1$: fissato $h \in \N$, $x^{(n)}_h \conv[n] 0$ ed inoltre le norme delle $x^{(n)}$ sono uniformemente limitate, dunque $x^{(n)} \weakconv 0$.
	Nel caso $p = 1$ invece, il limite candidato è $0$. Se $x^{(n)} \weakconv 0$, allora anche $x^{(n)} \conv 0$, ma ciò vorrebbe dire che $\|x^{(n)}\| \conv 0$, assurdo.
\end{exercise}

\begin{exercise}
	Sia $\{x^{(n)}\}_{n \in \N}$ in $\ell^2$, così definita:
	\begin{equation*}
		x^{(n)}_h = \begin{cases}
			\frac1h & h \neq n\\
			1 & h=n
		\end{cases}
	\end{equation*}
	Provare che $x^{(n)} \weakconv \{1/h\}_{h \in \N}$ ma non fortemente.
\end{exercise}

\begin{exercise}
	Dimostrare che $(E, \sigma(E,E'))$ è uno spazio vettoriale topologico.

	\textbf{Svolgimento}.
	Consideriamo $(x_1,x_2) \in E \times E$, e fissiamo un intorno di base di $x_1 + x_2$:
	\begin{equation*}
		U = \{x \in E \suchthat |f_i(x-(x_1+x_2))| < \varepsilon, \ \text{p.o. $i \in F$}\}, \ \text{$F$ finito, $f_i \in E'$ per ogni $i \in F$.}
	\end{equation*}
	Se troviamo un intorno di $(x_1,x_2)$ mandato in $U$, ho dimostrato la continuità. Fissiamo $F$ e le $f_i$, e definiamo
	\begin{eqalign*}
		U_1 &= \{x \in E \suchthat |f_i(x-x_1)| < \varepsilon/2, \ \text{per ogni $i \in F$}\},\\
		U_2 &= \{x \in E \suchthat |f_i(x-x_2)| < \varepsilon/2, \ \text{per ogni $i \in F$}\}.
	\end{eqalign*}
	Chiaramente $U_1 \ni x_1$, $U_2 \ni x_2$ e dunque $U_1 \times U_2 \ni (x_1,x_2)$. Ora siano $y_1 \in U_1$ e $y_2 \in U_2$, e verifichiamo che $y_1+y_2 \in U$:
	\begin{equation*}
		|f_i(y_1+y_2-(x_1+x_2))| < \varepsilon/2 + \varepsilon/2 = \varepsilon.
	\end{equation*}
\end{exercise}

L'analisi funzionale è storicamente affrontata in due diversi maniere: tramite gli spazi normati, seguendo Schwarz \cite{dunford1958linear,dunford1965linear}, o tramite gli spazi vettoriali topologici, seguendo Grothendieck \cite{grothendieck1973topological}.

\section{Topologia debole$^*$}
Ricordiamo che dato uno spazio normato, la funzione
\begin{eqalign}
	J : E &\longto E''\\
	x &\longmapsto (f \mapsto f(x))
\end{eqalign}
è la mappa di immersione di $E$ nel suo biduale. Come corollario al teorema di Hahn--Banach (Corollario~\ref{cor:equiv_norm}), abbiamo dimostrato che $J$ è un'isometria, ma in generale non è suriettiva, il che è un'ostruzione importante.

\begin{definition}
	Sia $E'$ il duale di uno spazio normato. La \defining{topologia debole$^*$} su $E'$ è la topologia iniziale associata a $\{J_x\}_{x \in E}$, immagine dell'immersione $E \into E''$.
\end{definition}

\begin{remark}
	Tale topologia è, in generale, meno fine della topologia debole di $E'$, in quanto $J$ è iniettiva ma in generale non suriettiva:
	\begin{equation*}
		\underbrace{\sigma(E',E)}_{\text{debole}^*} \ \subseteq\ \underbrace{\sigma(E',E'')}_{\text{debole}} \ \subseteq\ \|\cdot\|_{E'}.
	\end{equation*}
\end{remark}

\begin{lemma}
	\leavevmode
	\begin{enumerate}
		\item Fissato $f_0 \in E'$, una base di intorni aperti di $f_0$ rispetto alla topologia debole$^*$ è data da
		\begin{equation*}
			\{f \in E' \suchthat |(f-f_0)(x_i)| < \varepsilon, \text{ p.o. $i \in F$}\}, \quad \text{$F$ finito, $x_i \in E$ per ogni $i \in F$.}
		\end{equation*}
		\item La topologia debole$^*$ è di Hausdorff.
	\end{enumerate}
\end{lemma}
\begin{proof}
	\leavevmode
	\begin{enumerate}
		\item Segue dal fatto generale per le topologie iniziali.
		\item Siano $f_1 \neq f_2 \in E'$. Allora esiste almeno un punto $x \in E$ tale per cui $f_1(x) \neq f_2(x)$. Senza perdita di generalità, supponiamo che $f_1(x) < f_2(x)$. Sia allora $\alpha \in [f_1(x), f_2(x)]$ e definiamo
		\begin{eqalign*}
			U_1 &= \{f \in E' \suchthat |f(x)| < \alpha\} = J_x^{-1}(-\infty, \alpha)\\
			U_2 &= \{f \in E' \suchthat |f(x)| > \alpha\} = J_x^{-1}(\alpha, +\infty).
		\end{eqalign*}
		Questi sono aperti perchè controimmagini di aperti rispetto a mappe di valutazione. Inoltre $f_1 \in U_1$, $f_2 \in U_2$, $U_1 \cap U_2 = \varnothing$, e dunque la tesi è provata.
	\end{enumerate}
\end{proof}

Diremo che $C \subseteq E'$ è \defining{debolmente$^*$ chiuso} se il suo complementare è aperto rispetto alla topologia debole$^*$.
Diciamo che $f_n$ converge debole$^*$ ad $f$, scritto $f_n \weakconv^* f$, se $f_n$ converge a $f$ nella topologia debole$^*$.

\begin{lemma}
	Sia $E$ uno spazio normato, $\{f_n\}_{n \in \N}$ successione di $E'$, $f \in E'$.
	\begin{enumerate}
		\item $f_n \weakconv^* f$ se e soltanto se $f_n(x) \conv f(x)$ per ogni $x \in E$.
		\item $f_n \conv f$ implica $f_n \weakconv f$ implica $f_n \weakconv^* f$.
		\item Se $E$ è di Banach e $f_n \weakconv^* f$, allora
		\begin{equation*}
			\sup_n \|f_n\|_{E'} < \infty \ \text{e}\ \|f\|_{E'} \leq \liminf_n \|f_n\|_{E'}.
		\end{equation*}
		\item Se $E$ è di Banach e $f_n \weakconv^* f$ e $x_n \conv x$ in $E$, allora $f_n(x_n) \conv f(x)$.
	\end{enumerate}
\end{lemma}
\begin{remark}
	Il punto $(1)$ ci dice che la convergenza debole$^*$ in $E'$ non è altro che la convergenza puntuale!
\end{remark}
\begin{proof}
	\leavevmode
	\begin{enumerate}
		\item Caso particolare del Teorema~\ref{th:weak_conv_and_cont}.
		\item Banale, per inclusione delle topologie.
		\item Caso particolare del Corollario~\ref{cor:banach_steinhaus_liminf} al teorema di Banach--Steinhaus, in quanto $\{f_n\}_{n \in \N}$ è una successione di operatori tra spazi di Banach puntualmente convergenti.
		\item Per disuguaglianza triangolare,
		\begin{eqalign*}
			|f_n(x_n) - f(x)| &\leq |f_n(x_n) - f_n(x)| + |f_n(x) - f(x)|\\
			&\leq \|f_n\|_{E'}\,\|x_n - x\|_E + |f_n(x) - f(x)|.
		\end{eqalign*}
		Il primo termine è infinitesimo per ipotesi sulla successione $\{x_n\}_{n \in \N}$ e siccome dal punto precedente sappiamo che $\|f_n\|_{E'}$ è uniformemente limitato; mentre il secondo è infinitesimo per ipotesi sulla successione $\{f_n\}_{n \in \N}$.
	\end{enumerate}
\end{proof}

\begin{example}
	Ricordiamo che si può identificare $\ell^\infty$ con $(\ell^1)'$ tramite $J_x : y \mapsto \sum_{n=1}^\infty x_n y_n$ suriettiva. Ogni elemento di $x \in \ell^\infty$ si può immaginare come $J_x \in (\ell^1)'$. Diremo allora che $x^{(n)} \weakconv^* x$ in $\ell^\infty$ se $J_{x^{(n)}} \weakconv^* J_x$, cioè se $\sum_{h=1}^\infty x^{(n)}_h y_h \conv \sum_{h=1}^\infty x_hy_h$ per ogni $y \in \ell^1$.
\end{example}

\begin{exercise}
	Provare che, per $\{x^{(n)}\}_{n \in \N}$ in $\ell^\infty$,
	\begin{equation*}
		x^{(n)} \weakconv^* x \sse \text{$x^{(n)}_h \conv x_h$ per ogni $h \in \N$ e $\sup_n\|x^{(n)}\|_\infty < \infty$}.
	\end{equation*}

	\textbf{Svolgimento}.
	\begin{description}
		\item[$({\Longrightarrow})$] Già dimostrata (e segue dal lemma precedente).
		\item[$(\implied)$] Sia $y \in \ell^1$, ed $M>0$ tale che $\sup_n \|x^{(n)}\|_{\ell^\infty} < M$.
		\begin{eqalign*}
			\left| \sum_{h=1}^\infty x^{(n)}_h y_h - \sum_{h=1}^\infty x_h y_h \right| &\leq \sum_{h=1}^N |x^{(n)}_h - x_h||y_h| + \sum_{h=N+1}^\infty |x^{(n)}_h - x_h||y_h|\\
			&\leq (\max_{h=1,\ldots,N}|x^{(n)}_h - x_h|)\|y\|_{\ell^1} + 2M \sum_{h=N+1}^\infty |y_h|.
		\end{eqalign*}
		Scelgo $N$ grande in modo che $\sum_{h=N+1}^\infty |y_h| < \varepsilon$ fissato (posso farlo perche $y \in \ell^1$ significa proprio che tale serie è convergente, quindi ha code infinitesime). A questo punto mandando $n \to \infty$, anche il primo termine svanisce in quanto abbiamo assunto $x^{(n)}_h \to x_h$ per ogni $h \in \N$.
	\end{description}
\end{exercise}

\begin{lemma}
\label{lemma:weaktop_eight}
	Sia $E$ normato.
	\begin{enumerate}
		\item Se $T : E' \to \R$ è lineare e debolmente$^*$ continua allora esiste $x \in E$ tale che $T=J_x$.
		\item La topologia debole e la topologia debole$^*$ coincidono se e solo se $J$ è suriettiva.
	\end{enumerate}
\end{lemma}
\begin{proof}
	\leavevmode
	\begin{enumerate}
		\item Omissis.
		\item \begin{description}
			\item[$(\Longrightarrow)$] Sia $T \in E''$. Per definizione, $T$ è debolmente continuo, dunque per ipotesi $T$ è anche debolmente continuo$^*$. Ma allora per il primo punto, $T= J_x$ per un qualche $x \in E$, cioè $T \in \im J$.
			\item[$(\implied)$] Ovvio.
		\end{description}
	\end{enumerate}
\end{proof}

\subsection{Il teorema di Banach--Alaouglu}
Ricordiamo:

\begin{theorem}[Tychonoff]
\label{th:tychonoff}
	Sia $\{X_i\}_{i \in I}$ una famiglia di spazi topologici.
	Allora gli $X_i$ sono compatti se e solo se $\prod_{i \in I} X_i$ è compatto.
\end{theorem}

\begin{theorem}[Banach--Alaouglu]
\label{th:banach_alaouglu}
	Sia $E$ uno spazio normato.
	Allora $\closure B_{E'}(0,1)$ è debolmente$^*$ compatto.
\end{theorem}
\begin{proof}
	Consideriamo lo spazio $\R^E = \prod_{x \in E} \R$, dotato della topologia prodotto.
	Chiamiamo $\Phi : \closure B_{E'}(0,1) \to \R^E$ la mappa che manda una funzione $f : E \to \R$ nella palla unitaria di $E'$ nella sua `rappresentazione' come $E$-pla. Vogliamo mostrare che tale mappa è un'immersione topologica rispetto alla topologia debole$^*$ su $E'$.
	\begin{enumerate}
		\item \textbf{$\Phi$ è continua}. Dobbiamo allora dimostrare che per ogni $x \in E$, $\pi_x \Phi$ è continua. Ma $\pi_x = J_x$, che è certamente debolmente$^*$ continua.
		\item \textbf{$\Phi^{-1}$, definita sull'immagine di $\Phi$, è continua}. Dobbiamo allora dimostrare che per ogni $x \in E$, $J_x \Phi^{-1}$ è continua. Ma questa è la proiezione $x$-esima, dunque continua rispetto alla topologia prodotto.
	\end{enumerate}
	Possiamo dunque dimostrare il teorema mostrando la compattezza di $\Phi(\closure B_{E'}(0,1))$. La strategia è di mostrare che tale insieme è l'intersezione di un chiuso $C$ ed un compatto $K$ in $\R^E$. Affermiamo che
	\begin{eqalign*}
		C &= \{f : E \to \R \suchthat \text{$f$ lineare}\},\\
		K &= \{f : E \to \R \suchthat |f(x)| \leq \|x\|_E, \text{per ogni $x \in E$}\}.
	\end{eqalign*}
	Il primo è chiuso perchè equalizzatore di chiusi, mentre il secondo è compatto perchè uguale al prodotto di compatti $\prod_{x \in E} [-\|x\|_E, \|x\|_E]$. Il teorema è provato perchè $C \cap K$ è esattamente l'insieme di tutte le funzioni lineari e continue di norma al più unitaria, cioè $\Phi(\closure B_{E'}(0,1))$.
\end{proof}

\begin{remark}
	Sostanzialmente, il teorema di Alaouglu mostra che la topologia debole$^*$ è la topologia prodotto, in quanto le valutazioni puntuali si comportano come proiezioni.
\end{remark}

\subsection{Spazi riflessivi, teorema di Kakutani}
\begin{definition}
	Si dice \defining{riflessivo} uno spazio normato $E$ per cui l'isometria canonica delle valutazioni $J : E \into E''$ è suriettiva.
\end{definition}

\begin{remark}
	Riflessivo implica Banach, poichè $E''$ è Banach e $J$ ne esibisce un omeomorfismo con $E$.
\end{remark}
\begin{remark}
	Dal Lemma~\ref{lemma:weaktop_eight}, $E$ è riflessivo se e solo se topologia debole e debole$^*$ coincidono.
\end{remark}

\begin{lemma}[Helly]
	Sia $E$ uno spazio normato, $\vec f : E \to \R^n$ tale che $f_1, \ldots, f_n \in E'$ e sia $\vec \alpha \in \R^n$.
	Allora
	\begin{equation*}
		\vec \alpha \in \closure{f(B_E(0,1))} \sse \text{per ogni $\vec \beta \in \R^n$},\ |\vec\alpha \cdot \vec\beta| \leq \|\vec f \cdot \vec \beta\|_{E'}.
	\end{equation*}
\end{lemma}
\begin{proof}
	\leavevmode
	\begin{description}
		\item[$(\Longrightarrow)$] Omissis.
		\item[$(\implied)$] Per assurdo, supponiamo che $\vec \alpha \notin \closure{f(B_E(0,1))}$. Allora possiamo usare il teorema di Hahn--Banach geometrico (Teorema~\ref{th:geom_hahn_banach}) per separare il compatto $\{\vec\alpha\}$ e il chiuso convesso $\closure{f(B_E(0,1))}$ con un iperpiano. In pratica, esistono $\vec\beta \in \R^n$ e $\gamma \in \R$ tali che
		\begin{equation*}
			\vec f(x) \cdot \vec \beta < \gamma < \vec \alpha \cdot \vec \beta, \qquad \text{per ogni $x \in B_E(0,1)$}.
		\end{equation*}
		D'altra parte, siccome la disuguaglianza vale uniformemente rispetto ad $x \in B_E(0,1)$, si ricava
		\begin{equation*}
			\sup_{\|x\| \leq 1} |\vec f(x) \cdot \vec \beta| < \gamma < \vec \alpha \cdot \vec \beta, \qquad \text{per ogni $x \in B_E(0,1)$}.
		\end{equation*}
		Ma $\sup_{\|x\| \leq 1} |\vec f(x) \cdot \vec \beta| = \|\vec f \cdot \vec \beta\|_{E'}$, e dunque la disuguaglianza ottenuta contraddice l'ipotesi.
	\end{description}
\end{proof}

\begin{lemma}[Goldstein]
	Sia $E$ normato.
	Allora $J(B_E(0,1))$ è debolmente$^*$ denso in $B_{E''}(0,1)$.
\end{lemma}
\begin{proof}
	Sia $\xi \in B_{E''}(0,1)$ e $V$ intorno debole$^*$ di $\xi$ del tipo
	\begin{equation*}
		V = \{ \eta \in E'' \suchthat |(\eta-\xi)(f_i)| < \varepsilon, \text{per ogni $i=1,\ldots,n$} \}, \quad f_1, \ldots, f_n \in E'.
	\end{equation*}
	Vogliamo allora dimostrare che $J(B_E(0,1)) \cap V \neq \varnothing$, cioè dobbiamo trovare un $x \in B_E(0,1)$ tale che $|(J_x - \xi)(f_i)| < \varepsilon$. Essendo $\varepsilon$ arbitrario, ciò equivale a provare che il vettore $\vec\alpha = \vec f(\xi)$ appartenga a $\closure{f(B_E(0,1))}$. Allora possiamo appoggiarci al lemma di Helly, e provare che per ogni $\vec\beta \in \R^n$, si abbia $|\vec\alpha \cdot \vec\beta| \leq \|\vec f \cdot \vec\beta\|$.
	\begin{eqalign*}
		\left| \sum_{i=1}^n f_i(\xi)\beta_i \right| &= \left| \left(\sum_{i=1}^n f_i\beta_i \right)(\xi) \right|\\
		&\leq \|\xi\|_{E''}\, \|\vec f \cdot \vec\beta\|_{E'}\\
		&\leq \|\vec f \cdot \vec \beta\|_{E'} \comment{poichè $\|\xi\|_{E''} \leq 1$ per ipotesi.}
	\end{eqalign*}
\end{proof}

\begin{theorem}[Kakutani]
\label{th:kakutani}
	Sia $E$ normato.
	Allora $E$ è riflessivo se e solo se la palla unitaria chiusa $\closure B_E(0,1)$ è debolmente compatta.
\end{theorem}
\begin{proof}
	\leavevmode
	\begin{description}
		\item[$(\Longrightarrow)$] $E$ riflessivo significa che $J$ è un isomorfismo, dunque resta definita l'isometria continua (rispetto alle norme, e quindi anche rispetto alle topologie deboli) $J^{-1} : E'' \to E$.
		Banalmente $J^{-1}\closure B_{E''}(0,1) = \closure B_E(0,1)$, e il teorema di Alaouglu ci dà la compattezza debole$^*$ di $\closure B_{E''}(0,1)$, per cui se $J^{-1}$ fosse continua rispetto alla topologia debole$^*$ si $E''$, $\closure B_E(0,1)$ sarebbe compatto perchè immagine continua di un compatto (teorema di Weierstrass).

		Per fare ciò, debbo verificare che per ogni $f \in E'$, la mappa $f J^{-1} : (E'', \sigma(E'', E')) \to \R$ è continua. Essa agisce su un elemento $F \in E''$ per valutazione su $J^{-1}(F) \in E$, dunque è continua rispetto alla topologia debole$^*$.

		\item[$(\implied)$] Supponiamo che $\closure B_E(0,1)$ sia debolmente compatta in $E$.
		Sappiamo che $J:E \to E''$ è continua anche rispetto alle topologie deboli su $E$ ed $E''$. Notiamo, inoltre, che ciò resta vero se riduciamo la topologia di $E''$ alla topologia debole$^*$.
		Allora $J$ mappa compatti deboli di $E$ in compatti deboli$^*$ di $E''$, in particolare in chiusi deboli$^*$ (perchè la topologia debole$^*$ è Hausdorff).
		Dal lemma di Goldstein sappiamo inoltre che $J(\closure B_E(0,1))$ è densa in $\closure B_{E''}(0,1)$.
		Ma allora $J(\closure B_E(0,1)) = \closure{J(\closure B_E(0,1))} = \closure B_{E''}(0,1)$, che per linearità comporta $J(E) = E''$.
	\end{description}
\end{proof}

\begin{lemma}
	\leavevmode
	\begin{enumerate}
		\item Se $E$ è riflessivo e $M \leq E$ è un sottospazio chiuso, allora anche $M$ è riflessivo.
		\item Se $E$ è uno spazio vettoriale e $\|\cdot\|_1$, $\|\cdot\|_2$ sono norme equivalenti su $E$, allora $(E, \|\cdot\|_1)$ è riflessivo se e solo se $(E, \|\cdot\|_2)$ è riflessivo.
		\item Se $E$ ed $F$ sono spazi normati isometrici allora $E$ è riflessivo se e solo se $F$ è riflessivo.
	\end{enumerate}
\end{lemma}
\begin{proof}
	\leavevmode
	\begin{enumerate}
		\item Su $M$ possiamo mettere due `topologie deboli': la prima è la topologia debole associata alla topologia sottospazio di $M$, mentre l'altra è la topologia sottospazio di $M$ rispetto alla topologia debole di $E$.
		Affermiamo che in realtà queste due topologie coincidono.
		Consideriamo la mappa identica $I$ tra $(M, \|\cdot\|_E)$ e $(E, \|\cdot\|_E)$.
		Essa è continua rispetto alle norme, dunque anche rispetto alle topologie deboli. Segue che $\sigma(E,E')\vert_M \subseteq \sigma(M,M')$.
		Viceversa anche la mappa identica $(M, \sigma(E,E')\vert_M) \to (M, \sigma(M,M'))$ è continua.
		Infatti preso un elemento $f \in M'$, $fI = f : (M, \sigma(E,E')\vert_M) \to \R$ è continua, perchè per il teorema di Hahn--Banach $f$ ammette un'estensione $F \in E'$, debolmente continua su $E$.
		Allora $F\vert_M = f = fI$ è continua. Segue che $\sigma(M,M') \subseteq \sigma(E,E')$.

		Ora sapendo che $E$ è riflessivo, dal teorema di Kakutani sappiamo che $B_E(0,1)$ è debolmente compatta. D'altra parte $M$ è chiuso e convesso, quindi debolmente chiuso. Pertanto $B_E(0,1) \cap M$ è debolmente compatto in $E$, quindi debolmente compatto in $M$. Ma $B_E(0, 1) \cap M = B_M(0,1)$, quindi dal teorema di Kakutani si ottiene la tesi.

		\item Ovvio.

		\item Sia $T:E \to F$ un'isometria suriettiva. Allora $T$ e $T^{-1}$ sono debolmente continue, e dunque si scambiano palle unitarie conservando la compattezza. Dal teorema di Kakutani segue la tesi.
	\end{enumerate}
\end{proof}

\begin{theorem}
\label{th:weaktop_eleven}
	\leavevmode
	\begin{enumerate}
		\item Se $E$ è uno spazio normato riflessivo allora $E'$ è riflessivo,
		\item Se $E'$ è riflessivo ed $E$ è di Banach allora $E$ è riflessivo.
	\end{enumerate}
\end{theorem}
\begin{proof}
	\leavevmode
	\begin{enumerate}
		\item Se $E$ è riflessivo, le topologie debole e debole$^*$ su $E'$ coincidono. Per il teorema di Banach--Alaouglu (Teorema~\ref{th:banach_alaouglu}), la palla $B_{E'}(0,1)$ è debolmente$^*$ compatta, quindi anche debolmente compatta per ipotesi. Dal teorema di Kakutani segue la tesi.

		\item Dal punto precedente, $E''$ è riflessivo. Allora l'isometria $J:E \to E''$ ha immagine completa in $E''$, in particolare ha immagine chiusa, pertanto dal lemma precedente riflessiva. Ma $E \iso J(E)$, dunque $E$ è riflessivo.
	\end{enumerate}
\end{proof}

Moralmente, questo teorema vuol dire:

\begin{corollary}
	Uno spazio di Banach è riflessivo se e solo se lo è il suo duale.
\end{corollary}

\begin{theorem}
	Sia $E$ spazio di Banach riflessivo, $K \subseteq E$ convesso, chiuso e limitato.
	Allora $K$ è debolmente compatto, ed inoltre se $f:K \to (-\infty,+\infty]$ è convessa e inferiormente continua\footnote{Cioè continua rispetto alla topologia inferiore su $\R$.}, allora $f$ ammette un punto di minimo su $K$.
\end{theorem}
\begin{proof}
	Siccome $K$ è convesso e chiuso, sappiamo che è anche debolmente chiuso. La limitatezza di $K$ significa che $K \subseteq \closure B(0,R)$ per un qualche $R>0$. D'altra parte, dal teorema di Kakutani $B(0,R) = R\,B(0,1)$ è debolmente compatto, quindi $K$ è un chiuso in un compatto in uno spazio di Hausdorff, per cui $K$ è debolmente compatto.

	Ora ricordiamo che il Teorema~\ref{th:weaktop_seven} ci dice che $f$ convessa e inferiormente semicontinua è anche debolmente inferiormente semicontinua. Allora per il teorema di Weierstrass segue che $f$ ha un minimo su $K$.
\end{proof}

\begin{remark}
	Il teorema di Weierstrass applicato ad una funzione debolmente inferiormente continua non garantisce la presenza di un massimo! Infatti la giusta generalizzazione del teorema di Weierstrass dice che una funzione continua manda compatti in compatti, ma i compatti di $\R_{\inf}$ non sono più intervalli chiusi e limitati, ma gli insiemi inferiormente chiusi.
\end{remark}

\begin{remark}
	L'esistenza del minimo di $f$ si può garantire anche rimpiazzando l'ipotesi di limitatezza di $K$ con l'ipotesi che $f$ sia illimitata su $K$ ($\lim_{x \in K, \|x\| \to \infty} f(x) = +\infty$).
\end{remark}

\section{Spazi separabili}
\begin{definition}
	Uno spazio metrico $(X, d)$ si dice \defining{separabile} se ha un sottoinsieme denso numerabile.
\end{definition}

\begin{lemma}
	Sia $(X,d)$ uno spazio metrico separabile, $Y \subseteq X$.
	Allora $(Y, d\vert_Y)$ è separabile.
\end{lemma}
\begin{proof}
	Sia $S = \{x_n\}_{n \in \N}$ l'insieme denso e numerabile di $X$. Consideriamo, per ogni $n,m \in \N$, la sottofamiglia numerabile di palle aperte $B(x_n, 1/m)$ che intersecano $Y$ in qualche punto (è necessariamente non vuota, per densità di $S$). Per ognuna di queste palle, sia $y_{n,m} \in B(x_n, 1/m) \cap Y$. Affermiamo che $\{y_{n,m}\}_{n,m \in \N}$ sia numerabile e denso in $Y$. Infatti preso $y \in Y$ ed $m \in \N$, esiste $n \in \N$ tale che $y \in B(x_n, 1/m)$ per densità di $S$ in $X$. Quindi $B(x_n, 1/m)$ è una delle palle della sottofamiglia selezionata sopra, per cui esiste $y_{n,m}$ ed è tale che $d(y,y_{n,m}) < 2/m$.
\end{proof}

\begin{theorem}
\label{th:weaktop_thirteen}
	Sia $E$ spazio normato, e sia $E'$ separabile.
	Allora $E$ è separabile.
\end{theorem}
\begin{proof}
	Sia $\{f_n\}_{n \in \N}$ denso in $E'$. Ricordiamo che $\|f_n\|_{E'} = \sup_{\|x\|=1} |\langle f_n, x \rangle|$. Dunque per ogni $n \in \N$, esiste $x_n \in \E$ di norma unitaria tale per cui
	\begin{equation*}
	\label{eq:dense_set_ineq}
		|\langle f_n, x_n \rangle| \geq \frac12 \|f_n\|.
	\end{equation*}
	Sia $L = \R[x_n,\, n \in \N]$ lo spazio vettoriale reale generato dai punti $\{x_n\}_{n \in \N}$. Dimostriamo che $L$ è denso in $E$. Per il Corollario~\ref{cor:boundlin_four}, ciò equivale a dimostrare che per ogni $f \in E'$, $f\vert_L \equiv 0$ comporta $f \equiv  0$.

	Sia dunque $f \in E'$ che si annulla su $L$. Per ogni $\varepsilon > 0$ esiste $n \in \N$ tale che $\|f-f_n\| < \varepsilon$.
	Si ha
	\begin{eqalign*}
		\|f\| &\leq \|f-f_n\| + \|f_n\|\\
			&\leq \varepsilon + 2 |\langle f_n, x_n \rangle| \comment{per \eqref{eq:dense_set_ineq}}\\
			&= \varepsilon + 2 |\langle f_n - f, x_n \rangle| \comment{poichè $f(x_n) = 0$}\\
			&\leq \varepsilon + 2 \|f_n - f\|_{E'} \|x_n\|\\
			&\leq 3\varepsilon.
	\end{eqalign*}
	Dunque $\|f\|_{E'} = 0$, da cui $f \equiv 0$, come desiderato.

	Infine, lo spazio vettoriale razionale $\Q[x_n,\, n \in \N]$ è denso in $L$ e numerabile, provando la separabilità di $E$.
\end{proof}

\begin{remark}
	Il viceversa non vale: se $E$ è separabile, non è detto che $E'$ lo sia. Ad esempio $L^1$ è separabile, ma in generale $L^\infty$ no.
\end{remark}

\begin{corollary}
\label{cor:ref_sep_dual}
	Sia $E$ spazio di Banach.
	Allora $E$ è riflessivo e separabile se e solo se $E'$ è riflessivo e separabile.
\end{corollary}
\begin{proof}
	\leavevmode
	\begin{description}
		\item[$(\Longrightarrow)$] Se $E$ è riflessivo, $J:E \to E''$ è un'isometria suriettiva, pertanto $E''$ è separabile. Ma $E''$ è il duale di $E'$, per cui dal teorema precedente concludiamo che $E'$ è separabile.
		\item[$(\implied)$] Segue dai Teoremi~\ref{th:weaktop_eleven} e~\ref{th:weaktop_thirteen}.
	\end{description}
\end{proof}

Per parlare di compattezza per successioni bisogna essere in un contesto metrico. Tuttavia abbiamo visto che le topologie deboli facilmente perdono questa caratteristica. Pertanto indagheremo ora questa proprietà.

\begin{theorem}
\label{th:weaktop_fourteen}
	Sia $E$ normato e separabile, e sia $\{x_n\}_{n \in \N}$ denso in $\closure B_E(0,1)$.
	Definiamo su $\closure B_{E'}(0,1)$ la seguente distanza:
	\begin{equation*}
		d(f,g) = \sum_{n=1}^\infty \frac1{2^n} |\langle f-g, x_n\rangle|, \qquad \text{per ogni $f,g \in B_{E'}(0,1)$}.
	\end{equation*}
	Allora la topologia associata a questa metrica coincide con la topologia debole$^*$ sulla palla unitaria di $E'$.
	In particolare, $(\closure B_{E'}(0,1), \sigma(E',E))$ è metrizzabile.
\end{theorem}
\begin{proof}
	Omettiamo la semplice verifica del fatto che $d$ sia ben definita come metrica.
	Scegliamo $f_0 \in \closure B_{E'}(0,1)$. Vogliamo mostrare che ogni intorno metrico di $f_0$ contiene un intorno debole$^*$ di $f_0$, e viceversa.

	Nel primo caso, fissiamo la palla $B_d(f_0, r) \subseteq \closure B_{E'}(0,1)$.
	Definiamo la seguente famiglia di intorni deboli$^*$:
	\begin{equation*}
		V_{n,\varepsilon} = \{ f \in \closure B_{E'}(0,1) \suchthat |\langle f-f_0, x_i \rangle| < \varepsilon, \text{per ogni $i=1, \ldots,n$}\}.
	\end{equation*}
	Sia $f \in V_{n,\varepsilon}$, abbiamo
	\begin{eqalign*}
		d(f,f_0) &= \sum_{i=1}^\infty \frac1{2^i}|\langle f-f_0, x_i\rangle|\\
		&= \sum_{i=1}^n \frac1{2^i}\underbrace{|\langle f-f_0, x_i\rangle|}_{< \varepsilon} + \sum_{i=n+1}^\infty \frac1{2^i}\underbrace{|\langle f-f_0, x_i\rangle|}_{< \underbrace{\|f-f_0\|_{E'}}_{< \|f\|_{E'} + \|f_0\|_{E'} = 2}\underbrace{\|x_i\|_E}_{=1}}\\
		&\leq \varepsilon \sum_{i=1}^n \frac1{2^i} + 2 \sum_{i=n+1}^\infty \frac1{2^i}.
	\end{eqalign*}
	Si osservi ora che il primo termine è controllato in magnitudine da $\varepsilon$, mentre il secondo lo è da $n$.
	Dunque, scegliendo $\varepsilon$ sufficientemente piccolo ed $n$ sufficientemente grande, possiamo rendere entrambi i termini inferiori a $r/2$.
	Il corrispondente $V_{n,\varepsilon}$ sarà allora contenuto nella palla $B_d(f_0, r)$.

	Viceversa, prendiamo un intorno debole$^*$ $V$ di $f_0$:
	\begin{equation*}
		V = \{f \in \closure B_{E'}(0,1) \suchthat |\langle f-f_0, y_i \rangle| < \varepsilon, \text{per ogni $i=1,\ldots,n$}\}, \quad y_1, \ldots, y_n \in E.
	\end{equation*}
	Senza perdita di generalità, possiamo assumere che gli $y_i$ usati per definire $V$ siano normalizzati.
	Cerchiamo un $r > 0$ per cui la palla metrica di raggio $r$ e centro $f_0$ sia contenuta in $V$. Supponiamo che $f \in B_d(f_0, r)$, necessariamente $|\langle f-f_0, x_i \rangle| < 2^i \, r$ per ogni $i \in \N$.
	Inoltre, fissato $\delta > 0$, per densità degli $\{x_n\}_{n \in \N}$ esistono $x_{h_1}, \ldots, x_{h_n}$ tali che $\|y_i - x_{h_i}\| < \delta$ per ogni $i=1,\ldots,n$.
	Verifichiamo ora che $|\langle f-f_0, y_i \rangle| < \varepsilon$:
	\begin{eqalign*}
		|\langle f-f_0, y_i \rangle| &\leq |\langle f-f_0, y_i - x_{h_i} \rangle| + |\langle f-f_0, x_{h_i} \rangle|\\
		&\leq \|f-f_0\| \underbrace{\|y_i-x_{h_i}\|}_{< \delta} +\; r 2^{h_i}\\
		&< 2 \delta + r 2^{\max_i h_i}.
	\end{eqalign*}
	Scegliendo $r$ e $\delta$ sufficientemente piccoli ambo i termini possono essere resi inferiori a $\varepsilon/2$ e dunque la disuguaglianza è dimostrata.
\end{proof}

\begin{remark}
	Si può anche dimostrare che la separabilità è condizione necessaria, oltre che sufficiente, per la metrizzabilità della palla unitaria in topologia debole$^*$.
\end{remark}

Diamo ora un corollario al teorema di Banach--Alaouglu (Teorema~\ref{th:banach_alaouglu}), che ne è una riespressione in termini sequenziali (à là Banach):

\begin{corollary}
\label{cor:weaktop_three}
	Sia $E$ normato e separabile.
	Allora ogni successione limitata di $E'$ ammette una sottosuccessione debolmente$^*$ convergente.
\end{corollary}
\begin{proof}
	A meno di riscalamento, si può assumere che la successione $\{f_n\}_{n \in \N}$ in oggetto sia contenuta nella palla unitaria chiusa di $E'$. Per il teorema di Banach--Alaouglu, quest'ultima è debolmente$^*$ compatta e per il teorema precedente è metrizzabile rispetto alla topologia debole$^*$ (dacchè $E$ è separabile). Allora il teorema segue per fatti generali sulle successioni nei compatti di uno spazio metrico.
\end{proof}

\begin{theorem}
	Sia $E$ spazio di Banach riflessivo.
	Allora ogni successione limitata di $E$ ammette una sottosuccessione debolmente convergente.
\end{theorem}
\begin{proof}
	Sia $\{x_n\}_{n \in \N}$ una successione limitata.
	Si consideri il sottospazio $M = \R[x_n,\, n \in \N]$. Esso è chiuso in $E$, quindi Banach e riflessivo anch'esso, e banalmente separabile.
	Consideriamo l'isometria canonica $J : M \to M''$, intesa come mappa tra gli spazi deboli. Restringiamola alla palla unitaria chiusa di $M$, e corestringiamola all'immagine di tale palla.

	Affermiamo che tale immagine sia metrizzabile. Siccome $M'$ è riflessivo (Corollario~\ref{cor:ref_sep_dual}), la topologia debole di $M''$ è coincidente con la topologia debole$^*$.
	Inoltre $M'$ è separabile (ibid.), da cui deduciamo che $\closure B_{M''}(0,1)$ è metrizzabile.
	Ma tale palla è omeomorfa a $\closure B_M(0,1)$ in topologia debole, quindi dal teorema di Kakutani $\closure B_M(0,1)$ è compatta per riflessività di $M$.
	La tesi segue dunque per fatti generali sulle successioni nei compatti di uno spazio metrico.
\end{proof}
\begin{proof}[Dimostrazione alternativa]
	Sia $\{x_n\}_{n \in \N}$ una successione limitata.
	Si consideri il sottospazio $M = \R[x_n,\, n \in \N]$. Esso è chiuso in $E$, quindi Banach e riflessivo anch'esso, e banalmente separabile.
	Per il Corollario~\ref{cor:weaktop_three} dunque, $\{x_n\}_{n \in \N}$ ammette una sottosuccessione debolmente$^*$ convergente in $M'' \iso M$.
	D'altra parte, siccome $M'$ è riflessivo (Corollario~\ref{cor:ref_sep_dual}), la topologia debole$^*$ di $M''$ coincide con la topologia debole. Pertanto il teorema è provato
\end{proof}

In definitiva, se $E$ è di Banach:
\begin{eqalign*}
	\text{$E$ separabile} &\sse \text{$(\closure B_{E'}(0,1),\, \sigma(E', E))$ è metrizzabile},\\
	\text{$E'$ separabile} &\sse \text{$(\closure B_E(0,1),\, \sigma(E, E'))$ è metrizzabile}.
\end{eqalign*}

\section{Spazi uniformemente convessi}
Cerchiamo un criterio per stabilire la riflessività di uno spazio.

\begin{definition}
	Si dice che uno spazio normato $E$ è \defining{uniformemente convesso} se per ogni $\varepsilon > 0$ esiste $\delta > 0$ tale che per ogni $x,y \in E$ di norma al più unitaria, distanti al più $\varepsilon$, si ha
	\begin{equation*}
		\left\|\frac{x+y}2\right\| < 1-\delta.
	\end{equation*}
\end{definition}

Uno spazio uniformemente convesso è, moralmente, uno spazio che `ha le palle tonde'.
La condizione di uniforme convessità impone che il punto medio tra due punti della palla sia sufficientemente lontano dal bordo. Ciò non vale se le palle sono quadrate, ad esempio, come in norma $1$ o $\infty$.

La contronomiale di questa condizione è che se il punto medio di due punti è sufficientemente vicino al bordo, allora tali punti non possono essere troppo lontani tra di loro.

\begin{theorem}[Milman]
\label{th:milman}
	Ogni spazio di Banach uniformemente convesso è riflessivo.
\end{theorem}
\begin{proof}
	Omissis. Vedere \cite{brezis2010functional}.
\end{proof}

\begin{theorem}
\label{th:weaktop_seventeen}
	Sia $E$ spazio uniformemente convesso e $\{x_n\}_{n \in \N}$ una successione in $E$ debolmente convergente a $x \in E$. Si supponga inoltre che $\limsup_n \|x_n\| = \|x\|$.
	Allora $x_n \conv x$.
\end{theorem}
\begin{remark}
	In buona sostanza, il teorema dice che
	\begin{center}
		convergenza debole $+$ convergenza delle norme $\implies$ convergenza in norma.
	\end{center}
	Infatti si ricordi che $\|x\| \leq \liminf_n \|x_n\|$ per una successione debolmente convergente, quindi l'ipotesi è equivalente a dire che $\lim_n \|x_n\| = \|x\|$.
\end{remark}
\begin{proof}
	Se $x=0$, il teorema è immediatamente verificato.
	Supponiamo allora che $x \neq 0$ e, senza perdita di generalità, che $x_n$ sia sempre non nulla (poichè una qualsiasi successione le cui norme non sono infinitesime è definitivamente non nulla). Poniamo $y_n = x_n/\|x_n\|$, sia $y = x/\|x\|$. Chiaramente $y_n \weakconv y$, $y_n + y \weakconv 2y$ e $(y_n+y)/2 \weakconv y$. Si ha
	\begin{equation*}
		1 = \|y\| \leq \liminf_n \left\|\frac{y_n+y}2\right\|
			\leq \liminf_n \frac{\|y_n\|+\|y\|}2
			= 1
	\end{equation*}
	Segue che $\|(y_n+y)/2\| \conv 1$, cioè il punto medio di $y_n$ e $y$ sta convergendo al bordo. Per uniforme convessità allora, necessariamente $\|y_n - y\| \conv 0$ fortemente, cioè $x_n \conv x$.
\end{proof}
