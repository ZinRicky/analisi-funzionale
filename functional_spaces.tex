\chapter{Spazi funzionali}
\section{Riflessività, separabilità e dualità in $L^p$}
Sia $(\Omega, \M, \mu)$ uno spazio di misura. Ricordiamo che, per $p \in [1,\infty)$, definiamo
\begin{eqalign*}
	\|f\|_{L^p} &:= \left( \int |f|^p\,\de\mu \right)^{1/p}\!\!, \qquad f:\Omega \to \R,\\
	L^p(\Omega) &:= \{ f : \Omega \to \R \suchthat \|f\|_{L^p} < \infty \}
\end{eqalign*}
Mentre
\begin{eqalign*}
	\|f\|_\infty &:= \esssup_{x \in \Omega} |f(x)| = \inf \{C \in \R \suchthat \mu(|f| \leq C) = 1 \}, \qquad f:\Omega \to \R,\\
	L^\infty(\Omega) &:= \{ f : \Omega \to \R \suchthat \|f\|_\infty < \infty \}
\end{eqalign*}

Le norme definite sopra sono soltanto seminorme (cioè $\|f\| = 0 \nimplies f \equiv 0$) se non quozientiamo gli spazi $L^p(\Omega)$ per la relazione di uguaglianza quasi ovunque. Una volta fatto questo, otteniamo spazi di Banach.

Nel seguente, assumeremo che $(\Omega, \M, \mu)$ sia $\sigma$-finito, ossia che $\Omega$ sia l'unione numerabile di sottoinsiemi di misura finita.

\begin{theorem}[Disuguaglianza di Clarkson]
	Se $2 \leq p < \infty$, ed $f,g \in \L^p(\Omega)$, si ha:
	\begin{equation*}
		\left\|\frac{f+g}2 \right\|_{L^p}^p + \left\|\frac{f - g}2 \right\|_{L^p}^p \leq \frac{\|f\|_{L^p}^p + \|g\|_{L^p}^p}2,
	\end{equation*}
	mentre se $1 < p \leq 2$:
	\begin{equation*}
		\left\|\frac{f+g}2 \right\|_{L^p}^q + \left\|\frac{f - g}2 \right\|_{L^p}^q \leq \left( \frac{\|f\|_{L^p}^p + \|g\|_{L^p}^p}2 \right)^{1/p-1}
	\end{equation*}
	dove $q = p/(p-1)$ è il \defining{coniugato di H\"older} di $p$.
\end{theorem}
Dimostreremo solo la prima, essendo la seconda parecchio più sofisticata.
\begin{proof}[Dimostrazione della prima]
	Riconduciamo la disuguaglianza in oggetto all'analoga disuguaglianza numerica:
	\begin{equation*}
		\left\|\frac{a+b}2 \right\|_{L^p}^p + \left\|\frac{a - b}2 \right\|_{L^p}^p \leq \frac{\|a\|_{L^p}^p + \|b\|_{L^p}^p}2, \qquad \text{per ogni $a,b \in \R$}.
	\end{equation*}
	Questa si prova dalla disuguaglianza di Jensen per i reali\footnote{Si ottiene come caso particolare della disuguaglianza di Jensen (Teorema~\ref{th:jensen}) applicandola ad una successione di soli due elementi (cioè con $a_n = 0$ per $n>2$).}
	\begin{equation*}
		(A^p + B^p)^{1/p} \leq (A^2+B^2)^{1/2}, \qquad \text{per ogni $A,B \geq 0$}.
	\end{equation*}
	Si pone allora:
	\begin{equation*}
		A = \frac{a+b}2, \qquad B = \frac{a-b}2
	\end{equation*}
	e si ottiene
	\begin{equation*}
		\left| \frac{a+b}2 \right|^p + \left|\frac{a-b}2\right|^p \leq \left( \left| \frac{a+b}2 \right|^2 + \left|\frac{a-b}2\right|^2 \right)^{p/2}
	\end{equation*}
	Svolgendo i quadrati al secondo membro:
	\begin{equation*}
		\left| \frac{a+b}2 \right|^p + \left|\frac{a-b}2\right|^p \leq \left( \frac{a^2+b^2}2 \right)^{p/2}
	\end{equation*}
	Nel caso $p \geq 2$, la funzione $x \mapsto x^{p/2}$ è convessa, dunque
	\begin{equation*}
		\left| \frac{a+b}2 \right|^p + \left|\frac{a-b}2\right|^p \leq\frac{|a|^p + |b|^p}2.
	\end{equation*}
	Applicando questa puntualmente ad $f$ e $g$, ed integrando, si ottiene la disuguaglianza cercata.
\end{proof}

\begin{theorem}
	Se $1 < p < \infty$, allora $L^p(\Omega)$ è uniformemente convesso e quindi riflessivo.
\end{theorem}
\begin{proof}
	Siano $f,g \in L^p(\Omega)$ e supponiamo $\|f\|_{L^p}, \|g\|_{L^p} \leq 1$, $\|f-g\|_{L^p} < \varepsilon$ per $\varepsilon>0$ fissato.
	Sia $p \geq 2$: dalla prima disuguaglianza di Clarkson:
	\begin{equation*}
		\left\| \frac{f+g}2 \right\|
		\leq \left(\frac{\|f\|_{L^p}^p+\|g\|_{L^p}^p}2 - \left\| \frac{f-g}2\right\|_{L^p}^p \right)^{1/p}
		\leq \left( 1 - \left(\frac{\varepsilon}{2}\right)^p \right)^{1/p} = 1 - \underbrace{\frac1p \left(\frac{\varepsilon}{2}\right)^p + o(\varepsilon^p)}_{\delta}.
	\end{equation*}
	Per $1 < p \leq 2$ si usa la seconda disuguaglianza in maniera analoga.
\end{proof}

\begin{theorem}[di rappresentazione di Riesz]
\label{th:riesz_repr}
	Sia $1 \leq p < \infty$, e sia $q$ il suo coniugato di H\"older.
	Per ogni $\varphi \in L^p(\Omega)'$, esiste $u \in L^q(\Omega)$ tale che
	\begin{equation*}
		\langle \varphi, f \rangle = \int_\Omega uf\,\de\mu, \qquad \text{per ogni $f \in L^p(\Omega)$}.
	\end{equation*}
	Inoltre, $\|\varphi\|_{(L^p)'} = \|u\|_{L^q}$.
\end{theorem}
\begin{proof}
	La dimostrazione si distingue in due casi.

	\textbf{Caso $p \neq 1$}. Consideriamo $T:L^q \to (L^p)'$ che agisce su $u \in L^q$ come
	\begin{equation*}
		\langle Tu, f \rangle := \int_\Omega uf \,\de\mu.
	\end{equation*}
	Vale
	\begin{equation*}
		|\langle Tu, f \rangle| = \left| \int_\Omega uf\,\de\mu \right| \underset{\text{H\"o}}\leq \|u\|_{L^q} \|f\|_{L^p}
	\end{equation*}
	quindi $Tu$ è lineare e continuo ($T$ è ben definito), e si ha $\|Tu\|_{(L^p)'} \leq \|u\|_{L^q}$.

	Fissato ora $u \in L^q$, definiamo $u' := |u|^{q-2} \, u$. Si ha
	\begin{equation*}
		\|u'\|_{L^p}
		= \left( \int_\Omega |u|^{(q-2)p}\,|u|^p\,\de\mu \right)^{1/p}
		= \left( \int_\Omega |u|^{p+q-2p+p}\,\de\mu \right)^{1/p}
		= \|u\|_{L^q}^{q/p} < \infty.
	\end{equation*}
	Perciò
	\begin{equation*}
		\frac{|\langle Tu, u'\rangle|}{\|u'\|_{L^p}} = \frac{{\displaystyle \int_\Omega} |u|^{q-2}|u|^2\,\de\mu}{\|u\|_{L^q}^{q/p}} = \|u\|_{L^q}^{q(1-1/p)} = \|u\|_{L^q}.
	\end{equation*}
	Segue che $\|Tu\|_{(L^p)'} = \|u\|_{L^q}$, e quindi che $T$ è un'isometria.
	Per completezza di $L^q$ allora, $T(L^q)$ è chiuso in $(L^p)'$.

	Per provare che $T$ è suriettivo basta allora provarne la densità. Per fare ciò, prendiamo $F \in (L^p)''$ e supponiamo $F\vert_{T(L^q)} \equiv 0$.
	Dalla riflessività di $L^p$ sappiamo che $F = J_h$ per un certo $h \in L^p$. Ma allora
	\begin{equation*}
		0 = \langle F, Tu \rangle = \langle Tu, h \rangle, \qquad \text{per ogni $u \in L^q$}.
	\end{equation*}
	Definiamo $u = |h|^{p-2} h$. Si ha
	\begin{equation*}
		0 = \langle Tu, h \rangle = \int_\Omega uh\,\de\mu = \int_\Omega |h|^p\,\de\mu.
	\end{equation*}
	Segue che $h=0$, e dunque $F \equiv 0$, dimostrando la densità dell'immagine di $T$ in $(L^p)'$.

	\textbf{Caso $p=1$}. Ricordiamo che abbiamo assunto $\Omega$ essere $\sigma$-finito, quindi in particolare possiamo scrivere $\Omega = \bigcup_{n \in \N} \Omega_n$ dove, senza perdita di generalità, si ha $\Omega_n \subseteq \Omega_{n+1}$ e $\mu(\Omega) < \infty$.
	Sia inoltre $\theta \in L^2$, $\theta > 0$ e pari ad una costante $\varepsilon_n$ su ciascuno degli $\Omega_{n+1} \setminus \Omega_n$ [per esempio: $\varepsilon_n = \sqrt{1/(n^2 \mu(\Omega_{n+1}\setminus \Omega_n))}$].

	Osserviamo che data $\varphi \in (L^1)'$, essa induce un funzionale su $L^2$, viz. $\psi : L^2 \to \R$ definito come $\psi(f) = \langle \varphi, f \theta \rangle$.
	Infatti, per la disuguaglianza di H\"older, $f\theta \in L^1$. Inoltre $\psi$ è lineare (ovvio) e continua:
	\begin{equation}
	\label{eq:riesz_norm}
		|\langle \psi, f \rangle|=|\langle \varphi, f\theta\rangle| \leq \|\varphi\|_{(L^1)'} \|f\theta\|_1 \leq \|\varphi\|_{(L^1)'}\|\theta\|_2\|f\|_2.
	\end{equation}
	Dunque $\psi \in (L^2)'$, e per il caso $p=2$ (già dimostrato) del teorema di Riesz, esiste unica $v \in L^2$ tale che
	\begin{equation*}
		\psi(f) = \langle \varphi, f\theta \rangle = \int_\Omega fv \,\de\mu, \qquad \text{per ogni $f \in L^2$}.
	\end{equation*}
	A questo punto potremmo osservare che, se $g/\theta \in L^2$:
	\begin{equation}
	\label{eq:riesz_star}
		\langle \varphi, g \rangle = \langle \varphi, \frac{g}\theta\, \theta \rangle
		= \int_\Omega \frac{g}\theta\, v \,\de\mu
		= \int_\Omega g \left( \frac{v}\theta \right)\,\de\mu
		= \int_\Omega gu\,\de\mu, \quad \text{dove $u = v/\theta$}.
	\end{equation}
	Dunque $u$ è un naturale candidato per essere l'elemento di $L^\infty$ che cerchiamo. C'è da dimostrare quindi che
	\begin{enumerate}
		\item $u \in L^\infty$, cioè che $\|u\|_\infty \leq \|\varphi\|$, e
		\item $\langle \varphi, g \rangle = \int_\Omega gu\,\de\mu$ per ogni $g \in L^1$.
	\end{enumerate}
	Proviamo ora questi due claim.
	\begin{enumerate}
		\item Osserviamo che~\eqref{eq:riesz_star} vale ancora se al posto di $g$ uso $g' \ind_{\Omega_n}$ per $g' \in L^\infty$ perchè in quel caso $g' \ind_{\Omega_n} \in L^p$ per ogni $p$ e $\theta$ è limitata dal basso da una costante, quindi $g' \ind_{\Omega_n}/\theta \in L^2$. Dunque~\eqref{eq:riesz_star} si riscrive come
		\begin{equation}
		\label{eq:riesz_starstar}
			\langle \varphi, g \ind_{\Omega_n} \rangle = \int_\Omega g \ind_{\Omega_n} u\,\de\mu, \qquad \text{per ogni $g \in L^\infty$}.
		\end{equation}
		Fissiamo ora $C > \|\varphi\|$ e sia $A = \{\|u\| > C\}$. Proviamo che $\mu(A) = 0$. Posto $g := \ind_A \sgn u$, dalla~\eqref{eq:riesz_starstar} segue
		\begin{equation*}
			\langle \varphi, \ind_A \ind_{\Omega_n} \sgn u \rangle = \int_{A \cap \Omega_n} |u|\,\de\mu \geq C \mu(A \cap \Omega_n).
		\end{equation*}
		D'altra parte,
		\begin{equation*}
			\langle \varphi, \ind_A \ind_{\Omega_n} \sgn u \rangle \leq \|\varphi\| \|\ind_A \ind_{\Omega_n} \sgn u\|_1 = \|\varphi\| \mu(A \cap \Omega_n)
		\end{equation*}
		Per cui si ha
		\begin{equation*}
			C \mu(A \cap \Omega_n) \leq \langle \varphi, \ind_A \ind_{\Omega_n} \sgn u \rangle \leq \|\varphi\| \mu(A \cap \Omega_n).
		\end{equation*}
		Dunque o $\mu(A \cap \Omega_n) = 0$, oppure
		\begin{equation*}
			C \leq \langle \varphi, \ind_A \ind_{\Omega_n} \sgn u \rangle \leq \|\varphi\|,
		\end{equation*}
		contraddicendo l'ipotesi che $C > \|\varphi\|$. Siccome $\mu(A \cap \Omega_n) = 0$ per ogni $n \in \N$, $\|u\| < C$ quasi certamente, da cui concludiamo che $\|u\|_\infty \leq \|\varphi\|$, come dovevasi dimostrare.

		\item Useremo un procedimento di troncatura: definiamo, per ogni $n \in \N$:
		\begin{equation*}
			g_n(x) = \begin{cases}
				g(x) & |g(x)| \leq n\\
				n \frac{g(x)}{|g(x)|} & \text{altrimenti}
			\end{cases}
		\end{equation*}
		Chiaramente: $g_n \conv g$, e $g_n \leq g$, dunque $g_n \convin{L^1} g$. Inoltre $g_n \in L^\infty$, quindi possiamo usare la~\eqref{eq:riesz_starstar} per dire
		\begin{equation*}
			\langle \varphi, g_n \ind_{\Omega_n} \rangle = \int_\Omega g_n \ind_{\Omega_n} u \,\de\mu,
		\end{equation*}
		Si ha comunque $g_n \ind_n \convin{L^1} g$, perchè gli $\Omega_n$ invadono $\Omega$.
		Da ciò deduciamo che $\langle \varphi, g_n \ind_{\Omega_n} \rangle \conv \langle \varphi, g \rangle$ per continuità, e $\int_\Omega  g_n \ind_{\Omega_n} u \,\de\mu \conv \int_\Omega g u \,\de\mu$ siccome $u \in L^\infty$. Ma questo è il risultato cercato:
		\begin{equation*}
			\langle \varphi, g \rangle = \int_\Omega gu\,\de\mu.
		\end{equation*}
	\end{enumerate}
	Il teorema è provato poichè, come conseguenza ulteriore del secondo punto, $\|\varphi\|_{(L^1)'} \leq \|u\|_{L^\infty}$, e quindi $\|\varphi\|_{(L^1)'}=\|u\|_{L^\infty}$ come anticipato.
\end{proof}

\begin{remark}
	L'intuizione fondamentale nel caso $p=1$ è il passaggio~\eqref{eq:riesz_star}. Dopodichè devo solo ampliare l'integrabilità di $g/\theta$.
\end{remark}

\begin{theorem}
	A meno che $(\Omega, \M, \mu)$ non sia costituito da un numero finito di atomi, $L^1(\Omega)$ (e quindi $L^\infty(\Omega)$) non sono riflessivi.
\end{theorem}
\begin{proof}
	Ci sono due casi:
	\begin{enumerate}
		\item Esiste una successione decrescente di misurabili $\{\Omega_n\}_{n \in \N}$ tali che $\mu(\Omega_n) \neq 0$ ma $\lim_n \mu(\Omega_n) = 0$.

		\item Altrimenti una tale successione non esiste, cioè ogni successione decrescente di misurabili non negligibili ha limite non negligibile. Pertanto $\Omega$ è formato da una quantità numerabile di atomi, ossia insiemi con misura non nulla ma i cui sottoinsiemi misurabili hanno misura nulla o pari a quella dell'insieme stesso.
	\end{enumerate}

	Nel primo caso, si considera la successione:
	\begin{equation*}
		u_n = \frac1{\mu(\Omega_n)} \ind_{\Omega_n}, \qquad n \in \N.
	\end{equation*}
	Se $L^1$ fosse riflessivo, questa successione (che è limitata) convergerebbe debolmente ad $u \in L^1$ (a meno di passare ad una sottosuccessione). Ciò implicherebbe che
	\begin{equation*}
		\int_\Omega u_n \varphi \,\de\mu \conv \int_\Omega u \varphi\,\de\mu, \qquad \text{per ogni $\varphi \in L^\infty$}.
	\end{equation*}
	Ma fissato $j \in \N$, si ha:
	\begin{equation*}
		\int_\Omega u_n \ind_{\Omega_j} \,\de\mu \conv[n] \int_\Omega u \ind_{\Omega_j}\,\de\mu.
	\end{equation*}
	Siccome gli $\Omega_n$ sono decrescenti, per $n \geq j$ si ha $\Omega_j \supseteq \Omega_n$, dunque $u_n \ind_{\Omega_j} = \frac1{\mu(\Omega_n)} \ind_{\Omega_n}$, che integrato su tutto $\Omega$ fa $1$. Dunque
	\begin{equation*}
		\int_\Omega u \ind_{\Omega_j} \,\de\mu = 1.
	\end{equation*}
	Ma siccome questo vale per ogni $j \in \N$, e $\int_\Omega u \ind_{\Omega_j}\,\de\mu \conv[j] 0$, avremmo una contraddizione.

	Nel secondo caso possiamo, senza perdita di generalità, assumere che $\Omega$ sia l'unione disgiunta di una quantità numerabile di punti $\{a_n\}_{n \in \N}$.
	In tal caso, $L^1 = \ell^1$, e proviamo che $\ell^1$ non è riflessivo.
	Infatti se così fosse, la successione degli elementi naturali $\{e_n\}_{n \in \N}$ avrebbe una sottosuccessione debolmente convergente ad $u \in \ell^1$. Siccome $\vec 1 =(1,1, \ldots) \in \ell^\infty$, si ha che $e_n \cdot \vec 1 \conv u \cdot \vec 1$, ma ciò significa che
	\begin{equation*}
		1 \conv u \cdot \vec 1
	\end{equation*}
	e quindi $u \cdot \vec 1 = 1$.
	È pacifico però che $e_n \weakconv u$ implica la convergenza puntuale (siccome tra i funzionali di $(\ell^1)'$ vi sono anche le proiezioni), la quale ha come limite $\vec 0$. D'altro canto $0 \cdot \vec 1 \neq 1$.

	Rimane solo il caso in cui il numero di atomi è finito, nel quale è banale che $L^1$ è riflessivo (perchè finito-dimensionale).
\end{proof}

Consideriamo ora il caso $(\Omega, \M, \mu) = (\R^n, \L, m)$.

\begin{lemma}
\label{lemma:neg_sep_lemma}
	Sia $(X,d)$ metrico ove esiste un aperto partizionato da una famiglia più che numerabile di aperti.
	Allora $(X,d)$ non è separabile.
\end{lemma}
\begin{proof}
	Supponiamo per assurdo che $(X,d)$ sia separabile, cioè che esista $X'=\{x_n\}_{n \in \N}$ denso in $X$. Sia $\{A_i\}_{i \in I}$ la partizione in ipotesi.
	Per densità di $X'$, per ogni $i \in I$ esiste un $n_i \in \N$ tale che $x_{n_i} \in A_i$.
	D'altro canto ciò non è possibile, siccome $\card{I} > \card\N$ e tutti gli $A_i$ sono disgiunti e non vuoti.
\end{proof}

\begin{theorem}
	Sia $\Omega$ un aperto di $\R^n$. Se $1 \leq p < \infty$, allora $L^p(\Omega)$ è separabile. Se $p=\infty$ allora $L^\infty(\Omega)$ non è separabile.
\end{theorem}

\begin{remark}
	Si può leggere questo teorema come il fatto che le funzioni regolari, ad esempio lisce, non sono dense in $L^\infty(\Omega)$. Questo perchè la convergenza uniforme conserva la regolarità, mentre in $L^\infty$ molti elementi non sono affatto regolari.
	Questo ha come conseguenza che l'approssimazione regolare di una funzione $L^\infty$ è una questione delicata, ad esempio la mollificazione non converge in norme (ma solo debolmente$^*$).
\end{remark}

\begin{proof}
	Nel caso $1 \leq p < \infty$, basta dimostrare che $L^p(\R^n)$ è separabile, essendo che $L^p(\Omega) \leq L^p(\R^n)$ (immergo estendendo le funzioni di $\Omega$ con zero).
	Usando il fatto che $\Czero_c(\R^n)$ è denso in $L^p(\R^n)$, è possibile vedere che tutte le combinazioni lineari a coefficienti razionali di funzioni caratteristiche del tipo
	\begin{equation*}
		\ind_{\prod_{i=1}^N (a_i,b_i)}, \quad a_i \leq b_i \in \Q,
	\end{equation*}
	sono dense in $L^p$.

	Nel caso $p=\infty$, consideriamo la famiglia $\{W_i\}_{i \in I}$ di tutte le palle aperte contenute in $\Omega$.
	Per ogni $i \neq j$, si ha $W_i \Delta W_j \neq \varnothing$, da cui $\|\ind_{W_i} - \ind_{W_j}\|_\infty \geq 1$.
	Consideriamo allora la famiglia
	\begin{equation*}
		\{ B_{L^\infty(\Omega)}(\ind_{W_i}, 1/2) \suchthat i \in I\}.
	\end{equation*}
	Affermiamo che tale famiglia è fatta da aperti disgiunti e non vuoti.
	Banalmente le palle in oggetto sono aperte e non vuote. Per verificare la disgiunzione, supponiamo per assurdo che, per certi $i \neq j$, esista $f \in B_{L^\infty(\Omega)}(\ind_{W_i}, 1/2) \cap B_{L^\infty(\Omega)}(\ind_{W_j}, 1/2)$.
	Allora
	\begin{equation*}
			\| \ind_{W_i} - \ind_{W_j} \| \leq \|\ind_{W_i} - f\|_\infty + \|f-\ind_{W_j}\|_\infty < \frac12 + \frac12 = 1,
	\end{equation*}
	assurdo.
	Perciò dal Lemma~\ref{lemma:neg_sep_lemma} segue la tesi.
\end{proof}

\begin{remark}
	In generale, $L^\infty$ non è separabile a meno che non sia riflessivo \cite[103]{brezis2010functional}.
\end{remark}

\section{Compattezza}
\begin{definition}
	Sia $(X,d)$ uno spazio metrico, un insieme $A \subseteq X$ si dice \defining{totalmente limitato} se per ogni $\varepsilon > 0$ esiste una famiglia finita $B_1, \ldots, B_{n_\varepsilon}$ di palle aperte di raggio $\varepsilon$ che ricoprono $A$.
\end{definition}

\begin{remark}
	Ogni insieme totalmente limitato è limitato, ma in uno spazio normato di dimensione infinita non vale il viceversa (si pensi alla palla unitaria, Teorema~\ref{th:unit_ball_not_compact}).
\end{remark}

\begin{lemma}
	Sia $(X,d)$ metrico.
	Allora $X$ è totalmente limitato se e solo se da ogni successione si può estrarre una sottosuccessione di Cauchy.
\end{lemma}
\begin{proof}
	Facile. Se la successione ha immagine finita, ok. Altrimenti, per ogni $\varepsilon = 1/n$, ricopro $X$ con un numero finito di palle, per cui almeno una di queste contiene una coda della successione.
	Sceglo il termine $n$-esimo da una di queste palle con la coda, la considero il nuovo $X$, e procedo allo stesso modo.
	Chiaramente la successione estratta in questa maniera è di Cauchy.
	Viceversa, negando l'ipotesi otteniamo una rete di $\varepsilon$-palle infinita i cui centri non ammettono una sottosuccessione di Cauchy.
\end{proof}

\begin{theorem}[Heine--Borel per spazi metrici]
\label{th:heine_borel}
	Sia $(X,d$) metrico. Allora le seguenti sono equivalenti:
	\begin{enumerate}
		\item $(X,d)$ è compatto.
		\item Da ogni successione si può estrarre una sottosuccessione convergente.
		\item $(X,d)$ è completo e totalmente limitato.
	\end{enumerate}
\end{theorem}
\begin{proof}
	Omissis.
\end{proof}

\begin{corollary}
	Sia $(X,d)$ uno spazio metrico completo, $A \subseteq X$ è relativamente compatto\footnote{Cioè ha chiusura compatta.} se e solo se è totalmente limitato.
\end{corollary}
\begin{proof}
	\leavevmode
	\begin{description}
		\item[$(\Longrightarrow)$] Segue da Heine--Borel.
		\item[$(\implied)$] Se $A$ è totalmente limitato, lo è pure $\closure A$. Dunque anche questa implicazione segue da Heine--Borel.
	\end{description}
\end{proof}

\begin{corollary}
\label{cor:comp_implies_sep}
	Se $(X, d)$ è compatto allora $(X, d)$ è separabile.
\end{corollary}
\begin{proof}
	Si consideri la successione di ricoprimenti data dalle palle di raggio $q_n$, dove $\{q_n\}_{n \in \N}$ è un'enumerazione dei razionali. Ciascuno di questi ricoprimenti ammette un raffinamento finito.
	I centri di tutti questi raffinamenti formano un insieme denso e numerabile.
\end{proof}

Indichiamo con $\Czero_b(X)$ le funzioni continue e limitate su $(X,d)$. Si osservi che se $X$ è compatto, $\Czero_b(X) \equiv \Czero(X)$.

\begin{definition}
	Sia $(X,d)$ metrico, $E \subseteq X$.
	Sia $\mathcal F$ una famiglia di funzioni definite su $E$ a valori in $\R$.
	\begin{enumerate}
		\item $\mathcal F$ si dice \defining{puntualmente limitato} se per ogni $x \in E$, l'insieme $\mathcal F(x)$ è limitato.
		\item $\mathcal F$ si dice \defining{equilimitata} se $\mathcal F(E)$ è limitato.
		\item $\mathcal F$ si dice \defining{equicontinua} se le $\mathcal F$ sono continue, e il modulo di continuità è uniforme rispetto ai punti e alle funzioni:
		\begin{equation*}
			\forall \varepsilon > 0\ \exists \delta > 0\ \forall x,y \in E\ \forall f \in \mathcal F,\ (d(x,y) < \delta \implies |f(x) - f(y)| < \varepsilon).
		\end{equation*}
	\end{enumerate}
\end{definition}

\begin{lemma}
\label{lemma:funsp_three}
	Se $E$ è numerabile e $\mathcal F = \{f_n\}_{n \in \N}$ puntualmente limitata su $E$, allora esiste una sottosuccessione di $\mathcal F$ puntualmente convergente.
\end{lemma}
\begin{proof}
	Si procede per diagonalizzazione sulle successioni $\{f_n(x_k)\}_{n,k \in \N}$. Si estraggono successivamente successioni convergenti puntualmente, ottenendo la sottosuccessione $f_{1,1}, f_{2,2}, \ldots$ che converge puntualmente.
\end{proof}

\begin{theorem}[Ascoli--Arzelà]
	Sia $(X,d)$ spazio metrico compatto, sia $\mathcal F =\{f_n\}_{n \in \N}$ una famiglia di funzioni puntualmente limitata ed equicontinua.
	Allora
	\begin{enumerate}
		\item $\mathcal F$ è equilimitata.
		\item $\mathcal F$ ammette una sottosuccessione uniformemente convergente.
	\end{enumerate}
\end{theorem}
\begin{proof}
	Fissiamo $\varepsilon > 0$. Sia $\delta := \delta_\varepsilon > 0$ il modulo di equicontinuità di $\mathcal F$ associato a $\varepsilon$.
	Per compattezza, $X$ è totalmente limitato, dunque posso trovare, per ogni $\varepsilon > 0$, un ricoprimento finito (di cardinalità $n_\delta \in \N$) di $X$ costituito da $\delta$-palle i cui centri, posso assumere, cadono in $E$ denso numerabile (siccome compatto implica separabile, Corollario~\ref{cor:comp_implies_sep}).
	Allora si ha:
	\begin{enumerate}
		\item Per ogni $x \in X$ esiste $1 \leq i \leq n_\delta$ tale che $x \in B(x_i, \delta)$. Allora
		\begin{equation*}
			|f_n(x)| \leq |f_n(x) - f_n(x_1)| + |f_n(x_1)| \leq \varepsilon + \sup_{1 \leq i \leq n_\delta} |f_n(x_i)| = M \lneq \infty,
		\end{equation*}
		cioè $\mathcal F$ è equilimitata.

		\item Dal Lemma~\ref{lemma:funsp_three}, a meno di passare ad una sottosuccessione, $\{f_n\}_{n \in \N}$ converge puntualmente in $E$. D'altro canto, per qualsiasi $x \in X$ esiste $1 \leq i \leq n_\delta$ tale che $x \in B(x_i, \delta)$. Si mostra che $\{f_n\}_{n \in \N}$ è Cauchy: infatti presi $n,m \in \N$ sufficientemente grandi, si ha
		\begin{equation*}
			|f_n(x) - f_m(x)| \leq \underbrace{|f_n(x)-f_n(x_i)|}_{< \varepsilon\ \text{(per equicont.)}} + \underbrace{|f_n(x_i) - f_m(x_i)|}_{< \varepsilon\ \text{(per conv. pt.)}} + \underbrace{|f_m(x_i) - f_m(x)|}_{< \varepsilon\ \text{(per equicont.)}} < 3\varepsilon
		\end{equation*}
		Si noti che solo la stima centrale non è, a priori, uniforme rispetto ad $x \in X$.
		Tuttavia siccome essa dipende da un numero finito di punti (i centri $x_i$ delle $\delta$-palle), possiamo renderla uniforme scegliendo $n,m \in \N$ sufficientemente grandi da soddisfare la disuguaglianza per ogni $x_i$.
		In definitiva possiamo concludere che la successione è di Cauchy in $\Czero(X)$, ed essendo questo spazio completo concludiamo che $f_n \conv f \in X$ uniformemente, come desiderato.
	\end{enumerate}
\end{proof}

Il teorema appena dimostrato sarebbe da chiamare più propriamente `teorema di Ascoli'. Il contributo di Arzelà è in realtà nel seguente corollario:

\begin{corollary}%[Caratterizzazione dei compatti di $\Czero(X)$]
	Sia $(X,d)$ spazio metrico compatto e sia $A \subseteq \Czero(X)$.
	Allora $A$ è totalmente limitato in $(\Czero(X), \|\cdot\|_\infty)$ se e soltanto se $A$ è una famiglia equilimitata ed equicontinua.
\end{corollary}
% \begin{remark}
% 	Il teorema di Heine--Borel (Teorema~\ref{th:heine_borel}) ci dice che $A$ totalmente limitato in $\Czero(X)$ è compatto, poichè tale spazio è completo quando
% \end{remark}
\begin{proof}
	\leavevmode
	\begin{description}
		\item[$(\Longrightarrow)$] Dalla totale limitatezza, otteniamo che $A$ è limitato in $\Czero(X)$, quindi $A$ è equilimitata come famiglia funzionale.
		D'altra parte, per ogni $\varepsilon > 0$ esistono $f_1, \ldots, f_{n_\varepsilon} \in A$ tali che le palle $B(f_i, \varepsilon)$ ricoprono $A$. In particolare, comunque preso $f \in A$ esiste un $1 \leq i \leq n_\varepsilon$ tale che $\|f-f_i\|_\infty < \varepsilon$.
		Inoltre, siccome $X$ è compatto, ciascuna di queste $f_i$ è uniformemente continua con modulo di continuità $\delta_i$.
		Siano dunque $x, y \in X$ a distanza inferiore di $\delta := \min_{1 \leq i \leq n_\varepsilon} \delta_i$. Abbiamo:
		\begin{equation*}
			|f(x) - f(y)| \leq \underbrace{|f(x) - f_i(y)|}_{< \varepsilon\ \text{(per tot. lim.)}} + \underbrace{|f_i(x) - f_i(y)|}_{< \varepsilon\ \text{(per unif. cont.)}} + \underbrace{|f_i(y) - f(x)|}_{< \varepsilon\ \text{(per tot. lim.)}} < 3\varepsilon.
		\end{equation*}
		cioe $A$ è una famiglia equicontinua.

		\item[$(\implied)$] Dal teorema di Ascoli--Arzelà, data una successione in $A$ questa ammette un'estratta uniformemente convergente.
		Pertanto $A$ soddisfa una delle condizioni di compattezza negli spazi metrici (Teorema~\ref{th:heine_borel}), e quindi $A$ è totalmente limitata.
	\end{description}
\end{proof}

In definitiva, il teorema di Ascoli--Arzelà ed il suo corollario danno una caratterizzazione dei compatti di $\Czero(X)$ per $X$ metrico compatto.

Esiste un risultato simile per la caratterizzazione della compattezza in $L^p(\R^n)$, che però non dimostriamo:

\begin{theorem}
	Sia $p \in [1,\infty)$, sia $\{f_n\}_{n \in \N}$ una successione in $L^p(\R^n)$ tale che
	\begin{description}
		\item[Equilimitatezza.] Esiste $M>0$ che limita uniformemente le norme:
		\begin{equation*}
			\|f_n\|_{L^p} \leq M, \qquad \text{per ogni $n \in \N$}.
		\end{equation*}
		\item[Equicontinuità delle traslate.] Per ogni $\varepsilon > 0$, esiste $\delta > 0$ tale che
		\begin{equation*}
			\int_{\R^n} |f_n(x+h) - f_n(x)|^p \,\dx < \varepsilon, \qquad \text{per ogni $\|h\| < \delta$, $n \in \N$}.
		\end{equation*}
		\item[Equi-$p$-concentrazione.] Per ogni $\varepsilon > 0$, esiste $R > 0$ tale che
		\begin{equation*}
			\int_{\R^n \setminus B(0,R)} |f_n(x)|^p\,\dx < \varepsilon, \qquad \text{per ogni $n \in \N$}.
		\end{equation*}
	\end{description}
\end{theorem}
\begin{proof}
	Omissis. Vedere \cite{brezis2010functional}.
\end{proof}

\begin{corollary}[Caratterizzazione dei compatti di $L^p(\R^n)$]
	Sia $p \in [1,\infty)$, $A \subseteq L^p(\R^n)$.
	Allora $A$ è totalmente limitato se e solo se
	\begin{enumerate}
		\item $A$ è limitato.
		\item $A$ è equicontinuo rispetto alle traslate.
		\item $A$ è equi-$p$-concentrato.
	\end{enumerate}
\end{corollary}
