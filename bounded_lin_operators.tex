\chapter{Operatori lineari e continui}
Ricordo che uno spazio normato su $\R$ (o $\C$) è uno spazio vettoriale $E$ su $\R$ (o $\C$) munito di una norma, cioè una funzione
$$
	\| - \|_E : E \to [0,+\infty)
$$
tale che
\begin{enumerate}
	\item per ogni $x \in E$, $\|x\|_E = 0 \sse x=0$,
	\item per ogni $x \in E$, $\lambda \in \R$ (o $\C$), $\|\lambda x \|_E = |\lambda|\|x\|_E$,
	\item per ogni $x,y \in E$, $\|x + y\|_E \leq \|x\|_E + \|y\|_E$.
\end{enumerate}
La norma definisce una distanza
$$
	d_E(x,y) = \|x-y\|_E, \qquad \forall x, y \in E
$$
che rende $E$ uno spazio metrico: per ogni $x,y,z \in E$,
\begin{enumerate}
	\item $d_E(x,y) = 0 \sse x=y$,
	\item $d_E(x,y) = d_E(y,x)$,
	\item $d_E(x,y) + d_E(y,z) \leq d_E(x,z) + d_E(z,y)$.
\end{enumerate}
Se $(E, d_E)$ è completo come spazio metrico, $(E, \|-\|_E)$ si dice \defining{spazio di Banach}.

\begin{lemma}
\label{lemma:reverse_triang_ineq}
	In uno spazio normato, vale la disuguaglianza
	\begin{equation}
		\|x-y\| \geq \|x\|-\|y\|
	\end{equation}
\end{lemma}

\begin{theorem}
	Siano $E$, $F$ spazi normati e $T: E \to F$ una funzione lineare. Allora sono equivalenti le seguenti affermazioni:
	\begin{enumerate}
		\item $T$ è continua,
		\item $T$ è continua in un punto $x_0 \in E$,
		\item $T$ è continua in $0$,
		\item esiste $C > 0$ tale che
		$$
			\|Tx\|_F \leq C \|x\|_E, \qquad \forall x \in E.
		$$
	\end{enumerate}
	In tali casi si dice che $T$ è un \defining{operatore (lineare) limitato}.
\end{theorem}
\begin{proof}
	\leavevmode
	\begin{description}
		\item[$(1) \implies (2)$] Ovvio.
		\item[$(2) \implies (3)$] La continuità in $x_0$ significa che per ogni $\varepsilon > 0$ esiste $\delta > 0$ tale che
		$$
			\|x-x_0\|_E < \delta \implies \|Tx - Tx_0\|_F < \varepsilon.
		$$
		Posto $y = x-x_0$, ciò comporta che
		$$
			\|y\|_E < \delta \implies \|Ty\| = \|Tx - Tx_0\|_F < \varepsilon
		$$
		ossia $T$ è continuo in $0$ (siccome ogni vettore $y$ di $E$ si può scrivere nella forma $x-x_0$ per un opportuno $x \in E$).
		\item[$(3) \implies (4)$] Per ogni $\varepsilon > 0$ esiste $\delta > 0$ tale che
		$$
			\|y\|_E < \delta \implies \|Ty\|_F < \varepsilon.
		$$
		Sia $x \in E$, $x \neq 0$. Poniamo
		$$
			y = \frac{x}{\|x\|_E}\frac{\delta}2
		$$
		Si ha $\|Ty\|_F < \varepsilon$, ossia
		$$
			\frac{\delta}{2\|x\|_E} \|Tx\|_F < \varepsilon
		$$
		da cui
		$$
			\|Tx\|_F < \frac{2\varepsilon}{\delta} \|x\|_E, \qquad \forall x \in E.
		$$
		\item[$(4) \implies (1)$] Sia $x_0 \in E$ fissato. Per ogni $x \in E$, sappiamo che
		$$
			\|T(x-x_0)\|_F \leq C \|x-x_0\|_E
		$$
		per una qualche costante $C > 0$. Sia $\varepsilon > 0$, e poniamo $\delta := \varepsilon/C$. Se $\|x-x_0\| < \delta$ allora si ha
		$$
			\|Tx - Tx_0\|_F = \|T(x-x_0)\|_F \leq C\|x-x_0\|_E < \varepsilon
		$$
		cioè $T$ è continuo in $x_0$.
	\end{description}
\end{proof}

\begin{lemma}
	Siano $E$, $F$ spazi normati e $T:E \to F$ un operatore limitato. Allora:
	\begin{eqalign*}
		&\inf\{C > 0 \suchthat \forall x \in E,\; \|Tx\|_F \leq C\|x\|_E \}\\&\overset{(1)}= \sup_{x \neq 0} \frac{\|Tx\|_F}{\|x\|_E}\\
		&\overset{(2)}= \sup_{\|x\|_E=1} \|Tx\|_F\\
		&\overset{(3)}= \sup_{\|x\|_E \leq 1} \|Tx\|_F.
	\end{eqalign*}
\end{lemma}
\begin{proof}
	\leavevmode
	\begin{description}
		\item[$(1)$] Ovvio.
		\item[$(2)$] Vale
		$$
			\frac{\|Tx\|_F}{\|x\|_E} = \left\| \frac{Tx}{\|x\|_E} \right\|_F = \left\|T\left(\frac{x}{\|x\|_E}\right)\right\|_F
		$$
		da cui segue l'uguaglianza cercata.
		\item[$(3)$] È ovvio che $\sup_{\|x\|_E=1} \|Tx\|_F \leq \sup_{\|x\|_E \leq 1} \|Tx\|_F$. D'altra parte:
		$$
			\sup_{\|x\|_E \leq 1} \|Tx\|_F \leq \sup_{0 < \|x\|_E \leq 1} \frac{\|Tx\|_F}{\|x\|_E} \leq  \sup_{x \neq 0} \frac{\|Tx\|_F}{\|x\|_E} = \sup_{\|x\|_E = 1} \|Tx\|_F.
		$$
	\end{description}
\end{proof}

\begin{definition}
	Siano $E$ ed $F$ spazi normati.
	\begin{enumerate}
		\item Si pone
		$$
			\L(E, F) := \{ T:E \to F \suchthat \text{$T$ è lineare e continuo}\}.
		$$
		\item Se $F=\R$ (o $\C$), allora si definisce
		$$
			E' = \L(E, F)
		$$
		il \defining{duale topologico} di $E$.
		\item Si pone
		$$
			\|T\|_{\L(E,F)} := \|T\|_{E \to F} := \sup_{x \neq 0} \frac{\|Tx\|_F}{\|x\|_E}.
		$$
	\end{enumerate}
\end{definition}

\begin{exercise}
	Provare che $\|-\|_{E \to F}$ definisce una norma su $\L(E,F)$.
\end{exercise}

\begin{remark}
	È vero che
	$$
		\|Tx\|_F \leq \|T\|_{E \to F} \|x\|_E, \qquad \text{per ogni $x \in E$},
	$$
	però in generale il $\sup$ che definisce $\|T\|_{E \to F}$ non è un max.
\end{remark}

\begin{example}
\label{ex:unif_deriv}
	Siano $E = \Cone[a,b]$, $F = \Czero[a,b]$, per $-\infty < a < b < +\infty$, entrambi muniti della norma della convergenza uniforme:
	$$
		\|x\|_E := \|x\|_F := \|x\|_\infty := \max_{t \in [a,b]} |x(t)|.
	$$
	Definiamo
	\begin{eqalign}
		T : \Cone[a,b] &\longto \Czero[a,b]\\
		x &\longmapsto Tx := \der{x}{t}
	\end{eqalign}
	$T$ è lineare ma non è continuo. Infatti la successione
	$$
		x_n(t) = \left(\frac{t-a}{b-a}\right)^n, \qquad t \in [a,b]
	$$
	è tale per cui
	$$
		\|x_n\|_\infty = 1, \qquad \left\|\der{x_n}{t}\right\|_\infty = \max_{t \in [a,b]} \left| n \left(\frac{t-a}{b-a}\right)^{n-1} \frac1{b-a} \right| = \frac{n}{b-a};
	$$
	cioè la successione $\{x_n\}_{n \in \N}$ è limitata mentre la sua immagine $\{\der{x_n}t\}_{n \in \N}$ non lo è!
\end{example}

\begin{question*}
	Cosa fare per rendere continuo $T : x \mapsto \der{x}{t}$?
\end{question*}

È sufficiente usare la `giusta' norma su $\Cone$, cioè
$$
	\|x\|_E = \|x\|_{\Cone[a,b]} = \|x\|_\infty + \left\|\der{x}{t}\right\|_\infty, \qquad \text{per ogni $x \in E$}.
$$
Adesso è vero che
$$
	\|Tx\|_F = \left\|\der{x}{t}\right\|_\infty \leq \|x\|_\infty + \left\|\der{x}{t}\right\|_\infty = \|x\|_E
$$
ossia $T$ è limitato di norma $\|T\| \leq 1$.

In realtà la norma di $T$ è esattamente $1$. Infatti posto $x_n(t) = \exp(n(t-a))$, si ha
$$
	\frac{\|Tx_n\|_F}{\|x_n\|_E} = \frac{\left\|\der{x}{t}\right\|_\infty}{\|x_n\|_\infty + \left\|\der{x}{t}\right\|_\infty} = \frac{n\e^{n(b-a)}}{\e^{n(b-a)} + n\e^{n(b-a)}} \conv[n] 1.
$$
Si osservi che non esiste $x \in E$ non nullo tale che $\|Tx\|_F = \|x\|_E$.

\begin{remark}
	Per provare la discontinuità nell'Esempio~\ref{ex:unif_deriv} si sarebbe potuta usare anche la successione:
	$$
		x_n(t) = \frac{\sin (nt)}n, \qquad \text{per ogni $t \in [a,b]$}
	$$
	la quale converge uniformemente a $0$ ma le cui derivate $\der{x_n}t = \cos(nt) \not\conv[n] 0$.
\end{remark}

\begin{example}
	Sia $\R[x]$ lo spazio dei polinomi su $\R$, normato con la norma uniforme su $[0,1]$. La funzione
	\begin{eqalign}
		T: \R[x] &\longto \R\\
		p &\longmapsto Tp := p(5)
	\end{eqalign}
	è lineare ma non continua. Infatti posto
	$$
		p_n(t) = t^n
	$$
	abbiamo $\|p_n\|_\infty = 1$, ma $Tx_n = 5^n \conv[n] \infty$. Cosa sarebbe successo se al posto di $5$ avessimo usato $1/2$?
\end{example}
\begin{example}
	Sia $E=\Czero_c[0,1]$ lo spazio delle funzioni reali su $[0,1]$ continue e a supporto compatto, normato con la norma uniforme. Sia
	\begin{eqalign}
		T: E &\longto \R\\
		x &\longmapsto Tx := \int_0^1 x(t)\,\dt
	\end{eqalign}
	Essa è evidentemente lineare, ed è anche continua:
	$$
		|Tx\| \leq \int_0^1 |x(t)|\,\dt \leq \|x\|_\infty.
	$$
	Si ha $\|T\| \leq 1$, e affermiamo che vale l'uguaglianza.
	Definiamo la successione

	\begin{minipage}[c]{\textwidth}
		\begin{minipage}[c]{.48\textwidth}
			$$
				x_n(t) = \begin{dcases}
					0 & 0 \leq t \leq \frac1n\\
					nt-1 & \frac1n \leq t \leq \frac2n\\
					1 & \frac2n \leq t \leq 1-\frac2n\\
					-nt+n-1 & 1-\frac2n \leq t \leq 1-\frac1n\\
					0 & 1-\frac1n \leq t \leq 1
				\end{dcases}
			$$
		\end{minipage}%
		\begin{minipage}[c]{.48\textwidth}
			\begin{figure}[H]
				\centering
				\begin{tikzpicture}
					\node (image) at (0,0){\includegraphics[width=.75\textwidth]{figures/comp_supp_seq.pdf}};
				\end{tikzpicture}
			\end{figure}
		\end{minipage}
	\end{minipage}
	\vspace{3ex}

	Chiaramente $\|x_n\|_\infty = 1$ e
	$$
		|Tx_n| = \left| \int_0^1 |x_n(t)|\,\dt \right| \conv[n] 1
	$$
	perciò $\|T\|=1$. Non esiste $x \in E$ tale che $\|x\|_\infty < 1$ e $|Tx| =1$ perchè $\supp x \subsetneq [\alpha, \beta]$.
\end{example}

\begin{lemma}
	Se $F$ è spazio di Banach, allora $E \to F$ è spazio di Banach con la norma definita sopra.
\end{lemma}
\begin{proof}
	Sia $\{T_n\}_{n \in \N}$ una successione di Cauchy in $E \to F$, dunque tale che per ogni scelta di $\varepsilon > 0$ esiste $n_\varepsilon \in \N$ tale che
	\begin{equation*}
		\|T_n - T_m\|_{E \to F} < \varepsilon, \qquad \text{per ogni $n , m \geq n_\varepsilon$}.
	\end{equation*}
	Per definizione di $\|\cdot\|_{E \to F}$, ciò comporta che
	\begin{equation}
	\label{eq:cauchy_estimate}
		\forall x \in E, \quad \|T_nx - T_mx\|_F \leq \|T_n - T_m\|_{E \to F}\|x\|_E < \varepsilon\|x\|_E.
	\end{equation}
	Per cui la successione $\{T_nx\}_{n \in \N}$ è di Cauchy in $F$ (si usi, dato $\varepsilon > 0$, l'$n_{\varepsilon/\|x\|_E}$ di $\{T_n\}_{n \in \N}$).
	Siccome $F$ è Banach, questo significa che $T_n x$ converge ad un certo elemento di $F$, che indichiamo con $Tx$: la corrispondenza $x \mapsto Tx$ definisce un operatore lineare, che affermiamo essere anche continuo e limite di $\{T_n\}_{n \in \N}$.

	Infatti mandando $m \to \infty$ in~\eqref{eq:cauchy_estimate} si ottiene
	\begin{equation*}
		\|T_n x - T x\|_F \leq \|T_n - T\|_{E \to F}\|x\|_E \leq \varepsilon \|x\|_E.
	\end{equation*}
	Riarrangiando,
	\begin{equation*}
		\|Tx\|_F \leq (\varepsilon + \|T_n\|_{E \to F})\|x\|_E
	\end{equation*}
	da cui la continuità di $T$. Infine, per ogni $n \geq n_\varepsilon$,
	\begin{equation*}
		\|T_n x - Tx\|_F \leq \varepsilon\|x\| \implies \|T_n - T\|_{E \to F} \leq \varepsilon
	\end{equation*}
	e quindi $T_n \to T$ in $E \to F$.
\end{proof}

\section{Norme equivalenti}
Due norme si dicono \defining{equivalenti} se inducono la stessa topologia. Di fatto, due norme su uno spazio $E$ sono equivalenti se l'identità è un omeomorfismo tra $(E, \|\cdot\|_1)$ ed $(E, \|\cdot\|_2)$. Ma questo significa che esistono $C_1, C_2 > 0$ tali che per ogni $x \in E$,
\begin{equation*}
	\|x\|_1 \leq C_1 \|x\|_2, \qquad \|x\|_2 \leq C_2 \|x\|_1;
\end{equation*}
da cui segue
\begin{equation*}
	\frac1{C_1}\|x\|_1 \leq \|x\|_2 \leq C_2 \|x\|_1.
\end{equation*}

\begin{lemma}
	Siano $E$, $F$ spazi normati.
	\begin{enumerate}
		\item Se $T : \R^n \to F$ è lineare allora è continua.
		\item Se $E$ è uno spazio di dimensione finita e $T : E \to F$ è lineare, allora $T$ è continua.
		%\item Se $T : \R^n \to F$ è lineare e biettiva allora $T$ e $T^{-1}$ sono continue.
		\item In uno spazio normato di dimensione finita, tutte le norme sono equivalenti.
	\end{enumerate}
\end{lemma}
\begin{proof}
	\leavevmode
	\begin{enumerate}
		\item Usando Cauchy--Schwarz nell'ultima disuguaglianza:
		\begin{equation*}
			\|Tx\|_F = \|T(x^i e_i)\|_F \leq |x^i|\|Te_i\|_F \leq C \|x\|_{\R^n}.
		\end{equation*}
		\item Siccome $E$ ha dimensione finita, esiste un isomorfismo $L:\R^n \to E$ per un qualche $n \in \N$. Per il punto precedente, questo isomorfismo è anche continuo.
		Di nuovo per il punto precedente, anche $TL$ è lineare e continua, per cui $T = TLL^{-1}$ dev'essere continua.
		\item Segue dai punti precedenti, siccome l'identità di uno spazio vettoriale è lineare.
	\end{enumerate}
\end{proof}

\section{Teorema di Hahn--Banach (Athos)}
L'analisi funzionale poggia su quattro teoremi `pilastro', che possiamo pensare come i tre (più uno) moschettieri.

\begin{theorem}
\label{th:hahn_banach}
	Sia $E$ uno spazio vettoriale su $\R$ e $p : E \to \R$ un'applicazione tale che
	\begin{enumerate}
		\item $p(\lambda x) = \lambda p(x)$ per ogni $x \in E$, $\lambda > 0$,
		\item $p(x+y) \leq p(x) + p(y)$ per ogni $x,y \in E$.
	\end{enumerate}
	Sia inoltre $G \leq E$ un sottospazio vettoriale, e $g:G \to \R$ un'applicazione lineare tale che
	\begin{enumerate}[resume]
		\item $g(x) \leq p(x)$ per ogni $x \in G$.
	\end{enumerate}
	Allora esiste un'applicazione lineare $f : E \to \R$ che estende $g$ ed è dominata da $p$.
\end{theorem}
\begin{proof}
	Si definisce
	\begin{equation*}
		P = \{ h : \mathcal D(h) \subseteq E \to \R \suchthat \text{$\mathcal D(h)$ ss.\! vettoriale, $h$ estende $g$ ed è dominata da $p$ } \}.
	\end{equation*}
	Si ordini $P$ con la relazione `$h_1$ estende $h_2$'. $P$ è non vuoto perchè contiene $g$, ed ogni catena di $P$ ha un elemento massimale (è semplicemente dato dall'unione insiemistica, che conserva la linearità\footnote{È un colimite diretto}). Dimostrato questo, invochiamo il lemma di Zorn per ottenere $f : D(f) \to \R$.

	Per mostrare che $D(f) = E$, procediamo per assurdo.
	Supponiamo $x_0 \in E \setminus D(f)$, e definendo $D(\varphi) = D(f) + \langle x_0 \rangle$, poniamo
	\begin{eqalign*}
		\varphi : D(\varphi) &\longto \R\\
				x + tx_0 &\longmapsto f(x) + t \alpha
	\end{eqalign*}
	dove $\alpha$ è da determinare in maniera che $\varphi \in P$. Se riusciamo a fare questo, abbiamo costruito un'estensione propria di $f$ che contraddice la massimalità di $f$.
	Dunque imponiamo
	\begin{equation*}
		f(x) + t\alpha = \varphi(x+tx_0) \leq p(x+tx_0), \qquad \text{per ogni $x \in D(f)$, $t \in \R$}.
	\end{equation*}
	Per linearità di $f$, è sufficiente verificare la disuguaglianza per $|t|=1$ (oltre che per il caso banale $t=0$). Infatti si ha (quando $t > 0$)
	\begin{equation*}
		f(x) + t\alpha = t\left(\varphi\left(\frac{x}{t} + x_0\right)\right) \leq t p\left(\frac{x}{t} + x_0\right) = p(x + tx_0).
	\end{equation*}
	e analogamente nel caso negativo. Allora scegliamo
	\begin{equation*}
		\sup_{y \in D(f)} (f(y) - p(y-x_0)) \leq \alpha \leq \inf_{x \in D(f)} (p(x+x_0) - f(x)).
	\end{equation*}
	Ciò è possibile in quanto $p$ domina $f$ ed è subadditiva.
\end{proof}

\begin{corollary}
\label{cor:hahn_banach_ext}
	Sia $G$ un sottospazio vettoriale di $E$ e $g : G \to \R$ lineare e continua.
	Allora esiste $f \in E'$ che estende $g$ e tale che $\|f\|_{E'} = \|g\|_{G'}$.
\end{corollary}
\begin{proof}
	Applichiamo Hahn--Banach con $p = \|g\|_{G'} \|\cdot\|_E$, che è chiaramente omogenea e subadditiva. Otteniamo un'estensione globale $f$ tale che
	\begin{equation*}
		|f(x)| \leq |p(x)| = \|g\|_{G'}\|x\|_E, \qquad \text{per ogni $x \in E$}.
	\end{equation*}
	Quindi $\|f\|_{E'} \leq \|g\|_{G'}$.
	D'altra parte, siccome $f$ estende $g$, sicuramente $\|f\|_{E'} \geq \|g\|_{G'}$.
\end{proof}

\begin{corollary}
\label{cor:two_chap_1}
	Sia $x_0 \in E \setminus\{0\}$.
	Allora esiste $f_0 \in E'$ tale che
	\begin{equation*}
		\|f_0\|_{E'} = 1, \qquad f_0(x_0) = \|x_0\|_E.
	\end{equation*}
\end{corollary}
\begin{proof}
	Applichiamo il corollario precedente a $g(tx_0) = t\|x_0\|_E$ definita su $\langle x_0 \rangle$. Chiaramente otteniamo $f_0$ di norma unitaria, e siccome $x_0 \in \langle x_0 \rangle$, $f_0(x_0) = g(x_0) = \|x_0\|_E$.
\end{proof}

\begin{corollary}
	\label{cor:eval_norm}
		Per ogni $x \in E$, detto $J_x : E' \to \R$ l'operatore di valutazione su $x$, si ha
		\begin{equation*}
			\|x\|_E = \sup_{\|f\|_{E'} \leq 1} |f(x)| = \|J_x\|_{E''}
		\end{equation*}
		ed il sup è realizzato (quindi è un max).
	\end{corollary}
\begin{proof}
	Se $x=0$, lo statement è ovvio. Supponiamo allora che $x \neq 0$, e osserviamo che, per ogni $f \in E'$ di norma $\|f\|_{E'} \leq 1$,
	\begin{equation*}
		\|J_x\|_{E''} = |f(x)| \leq \|f\|_{E'}\|x\|_E \leq \|x\|_E,
	\end{equation*}
	per cui anche il sup di della quantità di sinistra è controllato dall'alto da $\|x\|_E$. D'altra parte, sappiamo dal corollario precedente che esiste una funzione $f_0$ tale che $f_0(x) = \|x\|_E$, e che dunque realizza il sup.
\end{proof}

\begin{remark}
	In generale, $\|f\|_{E'}$ non è realizzato da nessun $x$.
\end{remark}

\begin{lemma}
\label{lemma:zero_outside_G}
	Sia $E$ uno spazio normato, $G \leq E$ sottospazio. Sia $x_0 \in E \setminus G$ un punto a distanza strettamente positiva da $G$.
	Allora esiste $f \in E'$ tale che $f(x_0) \neq 0$ e $f\vert_G \equiv 0$.
\end{lemma}
\begin{proof}
	Sia $\delta = \dist(x_0,G) >0$. Definiamo
	\begin{eqalign*}
		f : G+ \langle x_0 \rangle &\longto \R\\
			x + \lambda x_0 &\longmapsto \lambda \delta.
	\end{eqalign*}
	Si verifica facilmente che $f$ è lineare, e continua:
	\begin{eqalign*}
		|f(x+\lambda x_0)| &= |\lambda \delta|\\
		&= |\lambda|\dist(x_0, G)\\
		&\leq |\lambda| \dist\left(x_0, -\frac{x}{\lambda}\right)\\
		&\leq |\lambda| \left\|x_0 + \frac{x}\lambda \right\|\\
		&= \|\lambda x_0 + x \|.
	\end{eqalign*}
	Allora posso applicare Hahn--Banach (Corollario~\ref{cor:hahn_banach_ext}) ed estendere $f$ a tutto $E$. È ovvio che questa estensione soddisfi $f\vert_G \equiv 0$ e $f(x_0) = \delta \neq 0$.
\end{proof}

\begin{corollary}
\label{cor:boundlin_four}
	Sia $E$ uno spazio normato, $G \leq E$ sottospazio.
	Allora $G$ è denso in $E$ se e solo se per ogni $f \in E'$ tale che $f\vert_G \equiv 0$, si ha $f\vert_E \equiv 0$.
\end{corollary}
\begin{proof}
	\leavevmode
	\begin{description}
		\item[$(\Longrightarrow)$] Ovvio.
		\item[$(\implied)$] Se per assurdo $G$ non fosse denso, esisterebbe un punto $x_0$ a distanza positiva da $\closure G$. Per il lemma precedente, ciò ci darebbe una funzione nulla su $G$ ma non al di fuori, in contraddizione con l'ipotesi.
	\end{description}
\end{proof}

\begin{remark}
	Il teorema di Hahn--Banach è vero anche per spazi vettoriali su $\C$.% Difatti ogni funzione $\C$-lineare a valori in $\C$ è scomponibile nella sua parte reale e compless, che sono a loro volta funzioni lineari reali, a cui applichiamo il teorema.
\end{remark}

\section{Teorema di Banach--Steinhaus (Porthos)}
\subsection{Lemma di Baire}
Sia $(X,d)$ metrico e $A \subseteq X$.

\begin{lemma}[Baire]
\label{lemma:baire}
	Sia $X$ metrico completo, e sia $\{X_n\}_{n \in \N}$ una successione di chiusi con interno vuoto.
	Allora anche l'unione degli $X_n$ ha interno vuoto.
\end{lemma}
\begin{remark}
	Il lemma di Baire si usa spesso nella sua forma contronomiale, cioè osservando che se un'unione numerabile di chiusi copre $X$, allora ne esiste almeno uno con interno non vuoto.
\end{remark}
\begin{proof}
	Dimostriamo l'affermazione `duale': ponendo $O_n = X_n^C$, tutti gli $O_n$ sono aperti densi, e vogliamo mostrare che pure la loro intersezione è densa.
	Sia dunque $B(x,r)$ una palla arbitraria\footnote{Consideriamo le palle aperte.}. Dimostreremo che la sua intersezione con $G = \bigcap_{n \in \N} O_n$ è non vuota.

	Si ponga $O_0 = X$, $x_0 = x$, $r_0 = 0$ e si definisca la successione di palle aperte:
	\begin{equation*}
		B(x_{n+1}, r_{n+1}) \subseteq B(x_n, r_n) \cap O_{n+1}, \quad r_{n+1} \leq \frac{r_n}2 \leq \frac{r}{2^{n+1}}
	\end{equation*}
	Ciò è possibile in quanto ciascun $O_n$ è denso, e $B(x_n, r_n) \cap O_{n+1}$ è aperto poichè intersezione di aperti.

	Notiamo che $\{x_n\}_{n \in \N}$ è di Cauchy: infatti per ogni $m > n$ si ha
	\begin{eqalign*}
		d(x_n, x_{n+1}) \leq \frac{r}{2^n}.
	\end{eqalign*}
	Dalla completezza di $X$ deduciamo allora che $x_n \conv x_\infty \in X$. Infine, $x_\infty \in \bigcap_{n \in \N} B(x_n,r_n) \subseteq \bigcap_{n \in \N} (B(x,r) \cap O_n) \subseteq B(x,r) \cap G$, e la dimostrazione è conclusa.
\end{proof}

\begin{exercise}
\label{ex:ptwise_eventually_null_seq}
	Sia $E$ spazio di Banach, $\{f_n\}_{n \in \N}$ successione in $E'$. Supponiamo che per ogni $x \in E$, esista $n_x \in \N$ tale che
	\begin{equation*}
		f_{n_x}(x) = 0.
	\end{equation*}
	Allora esiste $\bar n \in \N$ tale che $f_{\bar n \equiv 0}$.

	\textbf{Svolgimento}. L'ipotesi è tale per cui
	\begin{equation*}
		\bigcup_{n \in \N} \ker f_n = E
	\end{equation*}
	in quanto ogni punto di $E$ è in un qualche $\ker f_n$. Lemma di Baire: esiste $\ker f_{\bar n}$ con interno non vuoto, il che significa che contiene un'intera palla aperta.
	Tuttavia se una funzione lineare si annulla su una palla, allora si annulla ovunque, da cui la tesi.
\end{exercise}

\begin{counterexample}
	Se $(X, d)$ non è completo allora il lemma di Baire non vale, in generale. Ad esempio $(\Q, |\cdot|)$ non è completo ed è coperto da un'unione numerabile di chiusi, viz. i suoi punti, i quali hanno tutti interno vuoto quando invece $\interior \Q \neq \varnothing$.
\end{counterexample}

\subsection{Il teorema}
\begin{theorem}[Banach--Steinhaus]
	Sia $E$ uno spazio di Banach e $F$ uno spazio normato, si consideri una famiglia arbitraria $\{T_i\}_{i \in I}$ di operatori lineari e continui $E \to F$.
	Supponiamo che
	\begin{equation}
	\label{eq:bs_1}
		\sup_{i \in I}\|T_ix\|_F < \infty, \qquad \text{per ogni $x \in E$}
	\end{equation}
	Allora
	\begin{equation*}
		\sup_{i \in I} \|T_i\|_{E \to F} < \infty,
	\end{equation*}
	cioè esiste $C > 0$ tale che
	\begin{equation*}
		\|T_i x\|_F \leq C\|x\|_E, \qquad \text{per ogni $x \in E$, $i \in I$}.
	\end{equation*}
\end{theorem}
\begin{remark}
	Si tratta di un principio di uniforme limitatezza:
	\begin{equation*}
		\text{maggiorazioni puntuali} \implies \text{maggiorazioni uniformi}.
	\end{equation*}
\end{remark}
\begin{proof}
	Definiamo
	\begin{equation*}
		X_n = \{ x\in E \suchthat \|T_ix\| \leq n, \text{per ogni $i \in I$}\}, \qquad n \in \N.
	\end{equation*}
	Si tratta di una successione numerabile di chiusi che, per l'ipotesi~\eqref{eq:bs_1}, ricopre $E$. Dal lemma di Baire deduciamo almeno uno di questi insiemi, diciamo $X_{n_0}$, ha interno non vuoto, cioè contiene almeno una palla $B(x_0, r)$.
	Per definizione degli $X_n$, per i punti $x_0 + rz$ (si prende $\|z\|_E \leq 1$) di questa palla vale
	\begin{equation*}
		\|T_i(x_0 + rz)\|_F \leq n_0, \qquad \text{per ogni $i \in I$}.
	\end{equation*}
	Segue che
	\begin{eqalign*}
		n_0 &\geq \|T_i(x_0 + rz)\|_F\\
		&= \|T_i(x_0) + rT_i(z)\|_F\\
		&= \|-T_i(-x_0) + rT_i(z)\|_F\\
		&\geq r\|T_i(z)\|_F - \|T_i(-x_0)\|_F
	\end{eqalign*}
	da cui
	\begin{equation*}
		\|T_i(z)\|_F \leq \frac{n_0 + \|T_i(x_0)\|_F}{r} \leq \frac{n_0 +\sup_{i \in I} \|T_i(x_0)\|_F}{r}
	\end{equation*}
	che è finito per ipotesi.
\end{proof}

\begin{counterexample}
	Per avere un principio di uniforme limitatezza come il precedente, è cruciale la linearità: sia $T_n:\R \to \R$ una funzione come in figura.

	(...)

	È chiaro che $\sup_n |T_n x| < \infty$, siccome ogni punto è definitivamente nel luogo degli zeri di $T_n$. Tuttavia $\sup_n \sup_x |T_n x| = + \infty$.
\end{counterexample}

\begin{counterexample}
	Anche l'ipotesi di completezza su $E$ è necessaria. Sicuramente, in tal caso non potremmo appoggiarci al lemma di Baire nella dimostrazione, ma comunque non esiste un'altra maniera di dimostrare il teorema.
	Si consideri $E = \Czero_c(0,1)$ con la norma $\|\cdot\|_\infty$. Esso non è completo (esercizio). Si consideri la famiglia di operatori $T_n : E \to \R$ definita come $T_nf = n f(1/n)$. Essi sono lineari, continui, e puntualmente limitati siccome $\lim_n T_n f = 0$ se $f$ è supportata in $(0,1)$ ($1/n$ è definitivamente fuori dal supporto di $f$). Si ha
	\begin{equation*}
		\sup_{f \in E, f \neq 0} \frac{|T_n f|}{\|f\|_\infty} \leq \sup_{f \in E, f \neq 0} \frac{n|f(1/n)|}{\|f\|_\infty} \leq \sup_{f \in E, f \neq 0} \frac{n\|f\|_\infty}{\|f\|_\infty} = n.
	\end{equation*}
	Per vedere che $\|T_n\|_{E'} = n$, osserviamo che una funzione come in figura realizza proprio tale norma:

	\begin{figure}[H]
		\centering
		\begin{tikzpicture}
			\node (image) at (0,0){\includegraphics[width=.5\textwidth]{figures/comp_supp_seq.pdf}};
		\end{tikzpicture}
	\end{figure}
\end{counterexample}

\begin{exercise}
	Sia $E = \R[x]$ normato da
	\begin{equation*}
		\left\|\sum_{k=0}^n a_k\,x^k\right\|_E = \max_{k = 0, \ldots, n} |a_k|.
	\end{equation*}
	Provare che $(E, \|\cdot\|_E)$ non è completo.

	\textbf{Svolgimento}. La successione $p_n = x^n/n$ è di Cauchy, ma il suo limite non è un polinomio.
	Per provarlo, usiamo Banach--Steinhaus, considerando la successione $T_n : E \to \R$ delle derivazioni $n$-esime.
	La successione è definitivamente limitata puntualmente, perchè ogni polinomio viene ucciso da una derivazione sufficientemente alta, ma $T_n x^n = n \to \infty$, per cui non abbiamo uniforme limitatezza.
	Una ulteriore maniera di confutare la completezza di $\R[x]$ è applicando l'Esercizio~\ref{ex:ptwise_eventually_null_seq}
\end{exercise}

Vediamo ora un corollario importante a Banach--Steinhaus:

\begin{corollary}
\label{cor:banach_steinhaus_liminf}
	Sia $E$ di Banach ed $F$ normato. Consideriamo una successione $\{T_n\}_{n \in \N}$ di operatori lineari e continui $E \to F$ tali che $\lim_n T_n x \in F$ per ogni $x \in E$, cioè $\{T_n\}_{n \in \N}$ è puntualmente convergente ad un operatore $T: E \to F$.
	Allora l'operatore limite $T$ è lineare e continuo, e si ha
	\begin{equation*}
		\|T\|_{E \to F} \leq \liminf_n \|T_n\|_{E \to F}.
	\end{equation*}
\end{corollary}
\begin{proof}
	Siccome $\{T_n\}_{n \in \N}$ è puntualmente convergente, è anche puntualmente limitata. Dal teorema di Banach--Steinhaus segue allora che è anche uniformemente limitata. Passando quest'ultima proprietà al limite $n \to \infty$, si ha
	\begin{equation*}
		\|Tx\|_F \leq C\|x\|_E, \qquad \text{per ogni $x \in E$}
	\end{equation*}
	che testimonia che $T$ è continuo, oltre che lineare.

	Ora, per definizione
	\begin{equation*}
		\|T_nx\|_F \leq \|T_n\|_{E \to F}\|x\|_E,
	\end{equation*}
	da cui
	\begin{eqalign*}
		\|T x\|_F &= \lim_n \|T_n x\|_F\\
		&= \liminf_n \|T_n x\|_F\\
		&\leq \left(\liminf_n \|T_n\|_{E \to F}\right) \|x\|_E
	\end{eqalign*}
	che dimostra la tesi.
\end{proof}

\begin{remark}
	Usiamo il $\liminf$ perchè non sappiamo se $\|T_n\|_{E \to F}$ abbia limite (l'ultimissima disuguaglianza non sarebbe verificata, in pratica, non essendo ben definito il membro di destra), mentre il limite inferiore esiste sempre. Anche il limite superiore ha questa proprietà, ma ovviamente avrebbe dato una stima peggiore.
\end{remark}

\begin{corollary}
	Sia $G$ normato, sia $B \subseteq G$.
	Se per ogni $f \in G'$, $f(B)$ è limitato in $\R$, allora $B$ è limitato.
\end{corollary}
\begin{proof}
	Applichiamo Banach--Steinhaus a $G'$, che è completo dal momento che $\R$ lo è.
	Consideriamo gli operatori $\{J_b : G' \to \R\}_{b \in B}$, dati dalle valutazioni. Da ipotesi, tale famiglia è puntualmente limitata, e Banach--Steinhaus ci dà uniforme limitazione. D'altra parte, $\|J_b\|_{E'} = \|b\|_E$ (Corollario~\ref{cor:eval_norm}), da cui la tesi.
\end{proof}

\begin{corollary}
	Sia $G$ Banach, sia $B' \subseteq G'$.
	Se per ogni $x \in G$ l'insieme $B'(x) = \{f(x) \suchthat f \in B'\}$ è limitato, allora $B'$ è limitato.
\end{corollary}
\begin{proof}
	È solo un restatement di Banach--Steinhaus.
\end{proof}

\begin{remark}
	Capiamo quindi che Banach--Steinhaus si può leggere come un teorema riguardo alla dualità della nozione di limitatezza.
\end{remark}

\section{Spazi di successioni}
Una successione di numeri reali o complessi sarà indicata con $(x_n)_{n \in \N}$.
Se $1 \leq p < \infty$, definiamo
\begin{equation*}
	\ell^p = \{ (x_n)_{n \in \N} \suchthat {\textstyle \sum_{n=1}^\infty} |x_n|^p < \infty \}
\end{equation*}
Per $x \in \ell^p$, si definisce anche la norma $p$:
\begin{equation*}
	\|x\|_{\ell^p} = \left( \sum_{n=1}^\infty |x_n|^p \right)^{1/p}.
\end{equation*}
Nel caso $p=\infty$, si definisce
\begin{equation*}
	\ell^\infty = \{(x_n)_{n \in \N} \suchthat {\textstyle \sup_{n \in \N}} |x_n| < \infty\}
\end{equation*}
con norma $\infty$ data da
\begin{equation*}
	\|x\|_{\ell^\infty} = \sup_{n \in \N} |x_n|.
\end{equation*}
Definiamo anche
\begin{eqalign*}
	c &= \{ (x_n)_{n \in \N} \suchthat x_n \conv[n] L \in \R \},\\
	c_0 &= \{ (x_n)_{n \in \N} \suchthat x_n \conv[n] 0 \},\\
	c_{00} &= \{ \text{$(x_n)_{n \in \N}$ è definitivamente nulla}\} = \R[x].
\end{eqalign*}
Chiaramente $c_{00} \subsetneq c_0 \subsetneq c$.

\begin{exercise}
	Provare che $c_0$ e $c$ sono sottospazi chiusi di $\ell^\infty$, e quindi dedurre che $c_0$ e $c$, con la norma $\|\cdot\|_\infty$, sono spazi di Banach.
\end{exercise}

È comodo costruire intuizione su questi spazi notando che $\ell^p$ è lo spazio di Lebesgue $L^p(\N, \parts \N, \sharp)$, mentre $c$ è lo spazio $\Czero(\bar N)$ delle funzioni continue $f : \bar \N \to \R$, dove $\bar N = \N \cup \{\infty\}$ è preso con la metrica
\begin{equation*}
	d(n,m) = |\arctan n - \arctan m|, \qquad n, m \in \bar\N.
\end{equation*}

\begin{theorem}[Disuguaglianza di Jensen]
\label{th:jensen}
	Siano $p<q \in [1, \infty]$.
	Allora
	\begin{equation*}
		\|x\|_{\ell^q} \leq \|x\|_{\ell^p}, \text{per ogni $x \in \ell^p$.}
	\end{equation*}
	In particolare, se $1 \leq p \leq q \lneq \infty$, abbiamo una sequenza di inclusioni continue:
	\begin{equation*}
		\ell^p \into \ell^q \into c_0 \into c \into \ell^\infty.
	\end{equation*}
\end{theorem}
\begin{remark}
	La prima inclusione è data dal presente teorema. La seconda è espressione della condizione necessaria di Cauchy di convergenza delle serie. La terza è ovvia. La quarta è espressione del teorema di limitatezza delle successioni convergenti.
\end{remark}
\begin{proof}[Dimostrazione nel caso $q \neq \infty$]
	Sia $x \in \ell^p$, e senza perdita di generalità supponiamo $\|x\|_{\ell^p} = 1$, da cui deduciamo $|x_n| \leq 1$ per ogni $n \in \N$. Sicuramente allora, $|x_n|^q \leq |x_n|^p$, da cui la disuguaglianza.
\end{proof}

\begin{exercise}
\label{ex:c00_density}
	Provare che $c_{00}$ è denso in $c_0$ ed in $\ell^p$ per $1 \leq p < \infty$, ma non in $\ell^\infty$.
\end{exercise}
\begin{exercise}
	Sia $(x_n)_{n \in \N}$ una successione di numeri reali, provare che se $\sum_{n=1}^\infty x_n y_n$ converge per ogni $(y_n)_{n \in \N} \in \ell^2$, allora $(x_n)_{n \in \N} \in \ell^2$, e viceversa.

	\textbf{Svolgimento}. Per ogni $n \in \N$ fissato, consideriamo l'operatore $T_n : \ell^2 \to \R$ definito come $T_ny = \sum_{k=1}^n x_ky_k$ per $y \in \ell^2$. Ovviamente tali operatori sono lineari, e inoltre
	\begin{equation*}
		|T_ny| = \left| \sum_{k=1}^n x_k y_k \right| \underset{C-S}\leq \left( \sum_{k=1}^n |x_k|^2\right)^{1/2}\! \left( \sum_{k=1}^n |y_k|^2 \right)^{1/2}
	\end{equation*}
	dove l'uguaglianza vale se e solo se $y$ e $x$ sono proporzionali. Dunque la norma ${\ell^2}'$ di $T_n$ è sicuramente inferiore o uguale di $\left( \sum_{k=1}^n |x_k|^2\right)^{1/2} = \|x\|_{\ell^2} < \infty$, e siccome questo bound è realizzato da $y=x$, vale l'uguaglianza.
	In particolare, ogni $T_n$ è continuo.

	Osserviamo ora che $\{T_n\}_{n \in \N}$ è puntualmente limitata per l'ipotesi sulla convergenza di $\sum_{n=1}^\infty x_n y_n$. Allora possiamo usare Banach--Steinhaus per dedurre che $\{T_n\}_{n \in \N}$ è limitata in norma ${\ell^2}'$, cioè che
	\begin{equation*}
		\sup_n \|T_n\|_{{\ell^2}'} = \sup_n \left( \sum_{k=1}^n |x_k|^2 \right)^{1/2} = \|x\|_{\ell^2} < \infty.
	\end{equation*}

	Viceversa,
	\begin{equation*}
		\left| \sum_{n=1}^\infty x_n y_n \right| \underset{C-S}\leq \left( \sum_{n=1}^\infty |x_n|^2\right)^{1/2}\! \left( \sum_{n=1}^\infty |y_n|^2 \right)^{1/2} < \infty
	\end{equation*}
\end{exercise}

\begin{exercise}
	Sia $1 \leq p \leq \infty$ e sia $(x_n)_{n \in \N}$ una successione di numeri reali tale che $\sum_{n=1}^\infty x_ny_n$ converge per ogni $y \in \ell^q$, dove $q$ è il coniugato di $p$. Provare che $x \in \ell^p$. \emph{Suggerimento: usare la disuguaglianza di H\"older.}
\end{exercise}

\begin{exercise}
\label{ex:ellp_iso_ellq}
	Sia $1 \leq p < \infty$, $q$ coniugato di $p$, e
	\begin{eqalign*}
		\Phi : \ell^q &\longto (\ell^p)'\\
			x &\longmapsto (y \mapsto \sum_{k=1}^\infty x_k y_k)
	\end{eqalign*}
	Si dimostri che $\Phi$ è un'isometria suriettiva.

	\textbf{Svolgimento}. La buona definizione di $\Phi$ è oggetto della disuguaglianza di H\"older. Il fatto che $\Phi$ sia isometria è lasciato come esercizio.

	Fissato ora $y \in \ell^p$, scriviamolo così: $y=(y_k)_{k \in \N} = \sum_{k=1}^\infty y_k e_k$, dove $(e_k)_h = \delta_{hk}$ (warning: questo funziona solo per $p < \infty$). Sia ora $F \in (\ell^p)'$. Si ha, per continuità di $F$:
	\begin{equation*}
		F(y) = F({\textstyle \sum_{k=1}^\infty} y_k e_k) = \sum_{k=1}^\infty y_k F(e_k).
	\end{equation*}
	Dunque $F = \Phi(F(e))$, siccome dall'esercizio precedente la convergenza di $F(y)$ per ogni $y \in \ell^p$ implica $F(e) \in \ell^q$.
\end{exercise}

\begin{remark}
	Per $p=\infty$, il precedente non vale, in particolare l'isometria esiste ma non è suriettiva. La sua immagine è isometrica a $c_0'$.
\end{remark}

\section{Teorema della mappa aperta (Aramis)}
\begin{theorem}
	Siano $E$, $F$ spazi di Banach, e $T : E \to F$ lineare e continua. Se $T$ è suriettiva, allora mappa insiemi aperti in insiemi aperti.
\end{theorem}

\begin{example}
	Se $E = F = \R$, $T : E \to F$ è necessariamente della forma $Tx = Cx$, per $C \in \R$. Se è suriettiva, $C \neq 0$. Ora supponiamo che $A$ sia un aperto, e dunque unione numerabile di intervalli aperti $(a_n, b_n)$. È banale verificare che $T(A)$ è dato dall'unione degli intervalli $(Ca_n, Cb_n)$, dunque aperta anch'essa.
\end{example}

\begin{remark}
\label{rmk:suff_cond}
	Per dimostrare il teorema, è sufficiente dimostrare che esiste una costante $C > 0$ tale che $B_F(0, C) \subseteq T(B_E(0,1))$. Infatti sia $U \subseteq E$ aperto, $x_0 \in U$ e $y_0 = Tx_0$. Siccome $U$ è aperto, esiste un $r > 0$ per cui $x_0 + B_E(0, r) = B_E(x_0, r) \subseteq U$, da cui
	\begin{eqalign*}
		T(U) &\supseteq T(x_0 + B_E(0,r))\\
			&= Tx_0 + T(B_E(0,r))\\
			&= y_0 + r T(B_E(0,1))\\
			&\supseteq y_0 +rB_F(0,C)\\
			&=  B_F(y_0, rC).
	\end{eqalign*}
\end{remark}

Un importantissimo, almeno a livello pratico, corollario del teorema della mappa aperta, è il seguente fatto:

\begin{corollary}
\label{cor:inv_cont}
	Siano $E$, $F$ spazi di Banach e sia $T:E \to F$ lineare, continua e biettiva.
	Allora $T^{-1} : F \to E$ è continua.
\end{corollary}
\begin{proof}
	Siccome $T$ è biettiva, $T^{-1}$ è ben definita. Ovviamente è lineare, e per il teorema della mappa aperta è continua.
\end{proof}
\begin{proof}[Dimostrazione alternativa]
	Se $y \in F$ e $\|y\|_F < C$, allora $y=Tx$ per $x \in B_E(0,1)$, quindi $\|T^{-1} y \| < 1$ e dunque $T^{-1}$ è limitato, dunque continuo.
\end{proof}

\begin{exercise}
	Siano $E$, $F$ spazi di Banach. Sia $T:E \to F$ lineare, continua e iniettiva.
	Allora le seguenti affermazioni sono equivalenti:
	\begin{enumerate}
		\item La mappa $T^{-1} : \im T \to E$ è continua.
		\item L'insieme $\im T$ è chiuso.
		\item Esiste $D > 0$ tale che
		\begin{equation*}
			\|Tx\|_F \geq D\|x\|_E, \qquad \text{per ogni $x \in E$}.
		\end{equation*}
	\end{enumerate}

	\textbf{Svolgimento}.
	\begin{description}
		\item[$(1) \implies (2)$] $T\vert^{\im T}$ è bicontinua, dunque $T$ e $T^{-1}$ sono Lipschitz:
		\begin{eqalign*}
			\|Tx_1 - Tx_2\|_F &\leq C\|x_1 - x_2\|_E, \qquad \text{per ogni $x_1, x_2 \in E$},\\
			\|T^{-1}y_1 - T^{-1}y_2\|_E &\leq C\|y_1 - y_2\|_F, \qquad \text{per ogni $y_1, y_2 \in \im T$},\\
		\end{eqalign*}
		Dacchè le Lipschitz mandano successioni di Cauchy in successioni di Cauchy $\im T$ è completo, quindi chiuso.
		\item[$(2) \implies (1)$] Essendo $\im T$ chiuso è Banach, dunque per il teorema della mappa aperta (in realtà per il Corollario~\ref{cor:inv_cont}) $T^{-1}$ è continua.
		\item[$(1) \iff (3)$] Sia $y=Tx$, la continuità di $T^{-1}$ si ha se e soltanto se esiste $D > 0$ tale che
		\begin{equation*}
			\|T^{-1}y\|_E \leq D\|y\|_F
		\end{equation*}
		cioè l'equazione desiderata.
	\end{description}
\end{exercise}

\begin{corollary}
\label{cor:equiv_norm}
	Sia $E$ uno spazio vettoriale munito di due norme $\|\cdot\|_1$ e $\|\cdot\|_2$. Supponiamo sia $(E, \|\cdot\|_1)$ che $(E, \|\cdot\|_2)$ siano \emph{completi}, e che esista $C > 0$ tale che
	\begin{equation*}
		\|x\|_2 \leq C\|x\|_1, \qquad \text{per ogni $x \in E$}.
	\end{equation*}
	Allora esiste $D > 0$ tale che
	\begin{equation*}
		\|x\|_1 \leq D \|x\|_2, \qquad \text{per ogni $x \in E$}
	\end{equation*}
	cioè le norme sono equivalenti.
\end{corollary}
\begin{proof}
	Sia $I : (E, \|\cdot\|_1) \to (E, \|\cdot\|_2)$ la mappa identità. La disuguaglianza in ipotesi ammonta ad affermare che $I$ è continua, in quanto limitata. Per il teorema della mappa aperta concludiamo allora che anche l'inversa è continua, dunque c'è omeomorfismo degli spazi e quindi equivalenza delle norme.
\end{proof}

\begin{remark}
	L'ipotesi di completezza per il teorema della mappa aperta è fondamentale. Ad esempio, l'identità tra $\Czero[0,1]$ con $\|\cdot\|_\infty$ e $\|\cdot\|_{L^1}$ è lineare e continua, ma non bicontinua. Infatti
	\begin{equation*}
		\|I(u)\|_{L^1} = \int_0^1 |u(t)|\,\dt \leq \|u\|_\infty,
	\end{equation*}
	mentre nell'altro senso non funziona. Una successione che converge nel dominio ma non nel codominio è una qualsiasi successione con patologia in altezza (e.g. $u_n(x) = [-n^3(x-1/n^2)]_+$).
\end{remark}

Infatti il completamento di $\Czero[0,1]$ rispetto alla norma $\|\cdot\|_{L^1}$ è $L^1[0,1]$, il che spiega perchè l'integrale di Lebesgue sia così importante per la matematica.

\begin{example}
	Sia $E = \Czero[0,1]$ con $\|\cdot\|_\infty$ e $F = \{ u \in \Cone[0,1] \suchthat u(0) = 0\}$ con $\|\cdot\|_\infty$.
	La funzione $T:E \to F$ definita come
	\begin{equation*}
		Tu (x) = \int_0^x u(t)\,\dt
	\end{equation*}
	è ben definita (cioè $Tu$ è sempre continua e svanisce in $0$), inoltre è lineare, continua, iniettiva e suriettiva. Il suo inverso $T^{-1} : F \to E$ è dato dalla derivazione, che sappiamo non essere continua.
\end{example}

\begin{corollary}
	Sia $E$ uno spazio vettoriale dotato di due norme equivalenti.
	Allora la completezza di $E$ con una delle norme determina la completezza con l'altra.
\end{corollary}

\begin{remark}
	Sia $(E, \|\cdot\|_1)$ uno spazio di Banach a dimensione infinita.
	Si può dimostrare che esiste sempre una norma $\|\cdot\|_2$ tale che $(E, \|\cdot\|_2)$ è ancora di Banach ma $\|\cdot\|_1$ non è equivalente a $\|\cdot\|_2$, cioè `essere di Banach' non è una proprietà ma una struttura.
\end{remark}

\begin{proof}[Dimostrazione del teorema della mappa aperta]
	Ricordiamo che è sufficiente provare quanto osservato in~\ref{rmk:suff_cond}. La dimostrazione avviene in due passi.

	\textbf{Passo 1}. Dimostriamo che esiste $C>0$ tale che $B_F(0, 2C) \subseteq \closure{T(B_E(0, 1))}$.
	Poniamo $X_n = \closure{n T(B_E(0,1))} = \closure{T(B_E(0,n))}$. L'unione degli $X_n$ copre $\im T = F$, dunque dal lemma di Baire otteniamo un $X_{\bar n}$ di interno non vuoto, da cui concludiamo che
	\begin{equation*}
		\frac1{\bar n} \interior{T(B(0,\bar n))} = \interior{\closure{T(B(0,1))}} \neq \varnothing.
	\end{equation*}
	Quindi esiste una palla $B_F(y_0, r) \subseteq \closure{T(B_E(0,1))}$. Per simmetria di $B_E(0,1)$, deduciamo che $-y_0 \in \closure{T(B_E(0,1))}$, ossia che $-y_0 + B_F(y_0, r) = B_F(0, r) \subseteq \closure{T(B_E(0,1))}$. Segue che $C=r/2$ è la costante cercata.

	\textbf{Passo 2}. Proviamo la seguente implicazione:
	\begin{equation*}
		B_F(0,2C) \subseteq \closure{T(B_E(0,1))} \implies B_F(0, C) \subseteq T(B_E(0,1)).
	\end{equation*}
	Sia $y \in B_F(0,C)$. Dimostriamo che esiste $x \in B_E(0,1)$ tale che $Tx = y$. Sappiamo che $B_F(0, C) \subseteq \closure{T(B_E(0,1/2))}$.
	Siccome le palle $T(B_E(0,1/2^n))$ sono dense nella propria chiusura, possiamo definire la seguente successione:
	\begin{equation*}
		B_E(0,1/2^n) \ni z_n \ \text{tale che}\ \|\underbrace{y - {\textstyle \sum_{k=1}^n} Tz_k}_{\in B_F(0, C/2^n)}\|_F < \frac{C}{2^n}
	\end{equation*}
	Ricordiamo adesso che una caratterizzazione equivalente degli spazi di Banach è la seguente:
	\begin{center}
		Uno spazio è di Banach se e solo se ogni serie assolutamente convergente\\
		($\sum_{n=1}^\infty \|x_n\|_E < \infty$) è convergente.
	\end{center}
	Segue che $\sum_{n=1}^\infty z_n$ converge, giacchè $\sum_{n=1}^\infty \|z_n\|_E < \sum_{n=1}^\infty 2^{-n} < 1$ converge come una serie geometrica. Chiamiamo $x$ il punto di convergenza della serie, che ha norma inferiore a $1$.
	Per continuità di $T$ inoltre, $T({\textstyle \sum_{n=1}^\infty} z_n) \conv Tx$, mentre per linearità:
	\begin{equation*}
		\|y- {\textstyle \sum_{n=1}^\infty} Tz_n\|_F = \|y- Tx\|_F = 0,
	\end{equation*}
	da cui $y = Tx$ con $x \in B_E(0,1)$, come desiderato.
\end{proof}

\begin{remark}
	Si osservi che il primo passo della dimostrazione usa la completezza di $F$ (tramite il lemma di Baire), mentre il secondo usa quella di $E$.
\end{remark}

\section{Teorema del grafico chiuso (D'Artagnan)}
Premettiamo che dati due spazi $(E, \|\cdot\|_E)$ e $(E, \|\cdot\|_F)$, è definito lo spazio prodotto $E \times F$ normato da
\begin{equation*}
	\|(x,y)\|_{E \times F} = \|x\|_E + \|y\|_F, \qquad \text{per ogni $(x,y) \in E \times F$}.
\end{equation*}
In questa maniera la convergenza in $E \times F$ si può verificare per componenti, e se sia $E$ che $F$ sono di Banach allora anche $E \times F$ lo è.

\begin{lemma}
	Siano $E \times F$ spazi normati, $T: E \to F$ lineare e continua.
	Allora il grafico $G(T)$ di $T$ è un sottospazio chiuso di $E \times F$.
\end{lemma}
\begin{proof}
	Sia $(x_n, T(x_n)) \in G(T)$ convergente a $(x,y)$. Ma per continuità di $T$, $Tx_n$ converge a $Tx = y$, quindi $G(T)$ è chiuso.
\end{proof}

Questo ci permetterà di ottenere alcuni metateoremi per dedurre la continuità di certi operatori, ad esempio per l'inclusione di spazi, in presenza di certe norme `naturali'.

\begin{theorem}[del grafico chiuso]
	Siano $E$, $F$ spazi di Banach e $T : E \to F$ lineare. Se il grafico $G(T)$ è chiuso in $E \times F$, allora $T$ è continuo.
\end{theorem}
\begin{proof}
	Definiamo in $E$ una nuova norma $\|\cdot\|_1$ ponendo
	\begin{equation*}
		\|x\|_1 = \|x\|_E + \|Tx\|_F, \qquad \text{per ogni $x \in E$}.
	\end{equation*}
	Mostriamo che $(E, \|\cdot\|_1)$ è di Banach.
	Sia $\{x_n\}_{n \in \N}$ di Cauchy in $(E, \|\cdot\|_1)$, cioè
	\begin{equation*}
		\lim_{n,m} \|x_n - x_m\|_E = 0, \qquad \lim_{n,m} \|Tx_n - Tx_m\|_F = 0.
	\end{equation*}
	Allora $\{(x_n, Tx_n)\}_{n \in \N}$ è di Cauchy in $E \times F$. Siccome $E$ ed $F$ sono di Banach, anche il loro prodotto lo è e dunque $\{(x_n, Tx_n)\}_{n \in \N}$ converge ad un punto $(x, y) \in E \times F$. D'altra parte, siccome $G(T)$ è chiuso, $(x,y) = (x,Tx) \in G(T)$.
	Perciò possiamo concludere che
	\begin{equation*}
		\|x_n - x\|_1 = \|x_n - x\|_E + \|Tx_n - Tx\|_F \conv[n] 0
	\end{equation*}
	e dunque $(E, \|\cdot\|_1)$ è completo.

	Si noti ora che $E$ ha due strutture di Banach, e si ha banalmente $\|x\|_E \leq \|x\|_1$. Dunque per il Corollario~\ref{cor:equiv_norm}, le due norme sono equivalenti. Ma siccome $T$ è continuo per $\|\cdot\|_1$, questo prova che è continuo anche per $\|\cdot\|_E$.
\end{proof}

\begin{remark}
	Il teorema del grafico chiuso non vale senza completezza di $E$ ed $F$.
	Il controesempio è dato da $E = \Cone[a,b]$ con la norma sup, $\Czero$ con la norma sup, e $T$ dato dalla derivazione. Abbiamo già visto che $T$ è lineare ma non continuo (Esercizio~\ref{ex:unif_deriv}), tuttavia andiamo a mostrare che $G(T)$ è chiuso prendendo una successione $\{(x_n, Tx_n)\}_{n \in \N} \subseteq G(T)$. Siccome $x_n \conv x$ in norma sup significa che si ha convergenza uniforme, e si ha pure $Tx_n = \der{x_n}{t} \conv y$, per un teorema classico sulle successioni di funzioni $y = \der{x}{t}$, testimoniando così la chiusura di $G(T)$.
\end{remark}

Gli operatori a grafico chiuso generalizzano gli operatori continui, come esplorato da Von Neumann. Questo è utile per poter parlare di operatori differenziali densamente definiti, che appaiono ad esempio in meccanica quantistica.

	\begin{exercise}
	Sia $\Omega \subseteq \R^n$ un insieme misurabile alla Lebesgue (ad esempio aperto), tale che $L^p(\Omega) \subseteq L^q(\Omega)$ per certi $p,q \in [1, \infty]$.
	Allora esiste $C > 0$ tale che
	\begin{equation*}
		\|f\|_{L^q} \leq C\|f\|_{L^p}, \qquad \text{per ogni $f \in L^p(\Omega)$},
	\end{equation*}
	cioè l'inclusione di $L^p(\Omega)$ in $L^q(\Omega)$ è continua.

	\textbf{Svolgimento}. Sia $I$ tale inclusione, che è ovviamente ben definitia e lineare. Per il teorema del grafico chiuso, se dimostriamo che $G(I)$ è chiuso allora possiamo concludere che $I$ sia continuo.
	Si prenda $\{(f_n, f_n)\}_{n \in \N} \subseteq G(I)$ convergente a $(f,g) \in L^p(\Omega) \times L^q(\Omega)$. Dalla convergenza $f_n \underset{L^p}\conv f$, deduciamo che esiste una sottosuccessione $\{f_{n_h}\}_{h \in \N}$ che converge puntualmente quasi ovunque. Ma siccome $f_n \underset{L^q}\conv g$, allora esiste una sottosuccessione di $f_h$ che converge quasi ovunque a $g$, e dunque $f \equiv g$ quasi ovunque, provando che $(f,g) \in G(I)$.
\end{exercise}

\begin{remark}
	La dimostrazione di questo bound non è costruttiva, quindi non abbiamo informazioni sulla costante $C$, i.e.\ la norma dell'inclusione. Nonostante ciò, ci sono casi in cui abbiamo una stima concreta su $C$, ad esempio quando $\Omega$ ha misura finita e $p > q$, allora $L^p(\Omega) \into L^q(\Omega)$ e
	\begin{equation*}
		\|f\|_{L^q} \leq |\Omega|^{\frac1q - \frac1p}\|f\|_{L^p}, \qquad \text{per ogni $f \in L^p(\Omega)$}.
	\end{equation*}
	Inoltre tale costante è sharp, poichè $f=1$ realizza l'uguaglianza.
\end{remark}

\section{Somme dirette topologiche}
\begin{lemma}
	Sia $E$ uno spazio di Banach, siano $G$ ed $L$ sottospazi chiusi tali che
	\begin{equation*}
		G+L = \{ x + y \suchthat x \in G, y \in L\}
	\end{equation*}
	è chiuso in $E$.
	Allora esiste $C > 0$ tale che, per ogni $z \in G+L$, esistono $x \in G$, $y \in L$ per cui
	\begin{equation*}
		z = x+y, \ \text{e}\ \|x\| \leq C\|z\|, \ \|y\| \leq C\|z\|.
	\end{equation*}
\end{lemma}
\begin{remark}
	Il lemma dice che possiamo trovare delle proiezioni continue controllate dalla stessa costante.
\end{remark}
\begin{proof}
	Consideriamo la mappa somma $G \times L \to G+L$, dove gli spazi sono normati rispettivamente dalla norma canonica del prodotto e dalla norma di $E$.
	$T$ è lineare, e la disuguaglianza di Minkowski della norma ne testimonia la continuità. Inoltre $T$ è suriettiva, per definizione di $G+L$.
	Vale dunque il teorema della mappa aperta, cioè è possibile trovare una palla aperta $B_{G+L}(0, C) \subseteq T(B_{G \times L}(0, 1))$, che in pratica significa che ogni $w \in G+L$ che abbia norma inferiore a $C$ è esprimibile come $\tilde x + \tilde y$, per $\tilde x \in G$ ed $\tilde y \in L$ e $\|\tilde x\|+\|\tilde y\|<1$.
	Sia $z \in G+L$, $z \neq 0$. Definiamo $w=C/(2\|z\|) z$. Questo vettore cade nella palla $B_{G+L}(0, C)$, e quindi per la considerazione appena fatta
	\begin{equation*}
		\frac{C}{2 \|z\|}\, z = \tilde x + \tilde y
	\end{equation*}
	da cui $x = \frac{2\|z\|}C \tilde x$ e $y= \frac{2\|z\|}C \tilde y$, che verificano le disuguaglianze in tesi.
\end{proof}

In generale, se io ho due insiemi chiusi la loro somma non è chiusa!

\begin{counterexample}[La linearità è necessaria]
	Sia $G = \R \times \{0\}$ e $L = \{(x,1/x) \in \R^2 \suchthat x >0\}$. Entrambi sono chiusi in $\R^2$, ma non $G+L$ che è il semipiano superiore aperto inferiormente (perchè $1/x > 0$ sempre).
\end{counterexample}

\begin{counterexample}[La chiusura è necessaria]
	Sia $E= \ell^2 \times \ell^2$, $G = \{(x,0) \in E \suchthat x \in \ell^2\}$ e $L=\{(x,Tx) \in E \suchthat x \in \ell^2\}$ per $T: \ell^2 \to \ell^2$ che manda $\{x_n\}_{n \in \N}$ in $\{x_n/n\}_{n \in \N}$. $G$ è chiuso perchè sottospazio, $L$ è chiuso perchè $T$ lineare continuo, ma $G+L$ non è chiuso in quanto se così fosse, $M=\{(0, Tx) \suchthat x \in \ell^2\}$ sarebbe chiuso, da cui $N= \im T$ sarebbe chiuso, assurdo, perchè $N$ è denso e $T$ non è suriettivo.
\end{counterexample}

\begin{definition}
	Sia $E$ spazio di Banach, $G$, $L$ sottospazi vettoriali tali che $E=G \oplus L$. La somma $G \oplus L$ è \defining{topologica} se inoltre $G$ ed $L$ sono chiusi, e si chiama $G$ il \defining{supplementare topologico} di $L$ e viceversa.
\end{definition}

\begin{theorem}
\label{th:top_sum}
	Sia $E$ spazio di Banach, $G$, $L$ sottospazi vettoriali tali che $E=G \oplus L$.
	Allora tale somma è topologica se e solo se le due proiezioni naturali $p_G : E \to G$, $p_L : E \to L$ sono continue.
\end{theorem}
\begin{proof}
	La direzione diretta è il lemma precedente. D'altra parte, $\ker p_G = G$ e $\ker p_L = L$, e i nuclei di operatori continui sono chiusi.
\end{proof}

\begin{definition}
\label{def:projector}
	Sia $E$ uno spazio vettore, $G \leq E$ sottospazio.
	Un \defining{proiettore lineare di $E$ su $G$} è una funzione lineare $p:E \to G$ che restringe all'identità su $G$.
\end{definition}

\begin{lemma}
	Sia $E$ Banach, $G \leq E$ sottospazio chiuso.
	Allora $G$ ammette un supplementare topologico se e solo se esiste un proiettore lineare e continuo $p:E \to G$.
\end{lemma}
\begin{proof}
	Se esiste un supplementare topologico, il Teorema~\ref{th:top_sum} ci dà un proiettore su $G$ lineare e continuo.
	D'altra parte, se $p$ è un tale proiettore, $\ker p$ è un sottospazio chiuso di $E$. È evidente che $\ker p \cap G = \{0\}$ e $E = \ker p \oplus G$: se $x \in E$, $x - p(x) \in \ker p$, $p(x) \in G$ e $p(x) +(x- p(x)) = x$.
\end{proof}

\begin{remark}
	Il proiettore su $\ker p$ è dato da $1-p$.
\end{remark}

\begin{theorem}
	Sia $E$ Banach, sia $G$ sottospazio di $E$.
	\begin{enumerate}
		\item Se $\dim G$ è finita, allora $G$ ammette un supplementare topologico.
		\item Se $G$ è chiuso e $\operatorname{codim} G = \dim E/G$ è finita, allora $G$ ammette un supplementare topologico.
	\end{enumerate}
\end{theorem}
\begin{proof}
	\leavevmode
	\begin{enumerate}
		\item Sia $e_1, \ldots, e_n$ una base di $G$. Per Hahn--Banach, gli elementi della base duale (in $G'$) si estendono a elementi di $E'$, per cui possiamo definire un proiettore lineare e continuo su $G$ dato da
		\begin{equation*}
			p(x) = \sum_{i=1}^n e^*_i(x)\,e_1, \qquad \text{per ogni $x \in E$}
		\end{equation*}
		dove indichiamo con $e^*_i$ l'estensione ad $E$ del duale di $e_i$.
		\item Banale una volta osservato che $E = G \oplus E/G$.
	\end{enumerate}
\end{proof}
